%%%-------------------------------------------------------------------
%%% Preface
%%%-------------------------------------------------------------------
\chapter*{Введение}
%%--------------------------------------------------------------------
%% The road to the third edition
%%--------------------------------------------------------------------
\section*{Путь к третьему изданию}

Моё первое знакомство с \GNUmake{} произошло, когда я был студентом в
университете Беркли в 1979 году. Мне посчастливилось работать с
<<новейшим>> оборудованием: компьютером DEC PDP 11/70, имевшим 128
килобайт оперативной памяти и терминал ADM 3a <<glass tty>>,
управляемым операционной системой Berkeley \UNIX{}, обслуживавшей
помимо меня ещё 20 пользователей!  Вспоминается, как много времени
отнимал вход в систему - пять минут с момента ввода имени пользователя
до момента получения приглашения командного интерпретатора.

После окончания университета я снова стал пользователем \UNIX{} лишь в
1984 году.  На этот раз я был программистом в Исследовательском Центре
NASA им. Д.С. Эймса.  Мы с моими коллегами выпустили одну из первых
систем \UNIX{} на базе микрокомпьютеров 68000, которые имели мегабайт
оперативной памяти, работали под управлением \UNIX{} Version 7 и
поддерживали одновременную работу до шести пользователей. Моим
последним проектом в NASA была интерактивная система анализа данных со
спутника, написанная на языке C с использованием таких инструментов
как yacc/lex и, конечно же, \GNUmake{}.

В 1988 году я вернулся в университет и работал над проектом программы
моделирования геометрических фигур с помощью сплайнов. Система
состояла из примерно 120000 строк кода на языке C и размещалась в
двадцати (или около того) исполняемых файлах. Система собиралась при
помощи \Makefile{}-шаблонов, расширяемых вспомогательной программой
\utility{genmakefile} (близкой по духу программе \utility{imake}) в
обычные \Makefile{}'ы. \textit{genmakefile} осуществляла включение
файлов, условную компиляцию и специфическую логику управления исходным
кодом и деревьями каталогов бинарных файлов. В те дни все были
уверены: для того, чтобы \GNUmake{} был полноценным инструментом
сборки программного обеспечения, ему требуется подобная обёртка.
Несколькими годами ранее я открыл для себя проект GNU и их версию
\GNUmake{}.  Тогда меня посетила мысль, что программа-обёртка,
возможно, не является необходимостью. Я переделал систему сборки так,
чтобы она не использовала шаблоны и генератор. К моему огорчению, мне
пришлось поддерживать систему на протяжении последующих четырёх лет
(поведенческий шаблон, по глупости повторяемый мной по сей день).
Система сборки была портирована на пять различных \UNIX{}-подобных
операционных систем и включала в себя раздельные деревья каталогов
исходных и бинарных файлов, автоматизированные ночные сборки и
возможность создания разработчиками частичных рабочих копий (partial
checkouts) с автоматическим заполнением недостающих объектов.

Моя следующая интересная встреча с \GNUmake{} произошла в 1996 году
при работе над коммерческой CAD\hyp{}системой. Моей задачей был
перенос двух миллионов строк кода на языке \Cplusplus{} (и ещё
примерно 400 000 строк кода на языке Lisp) с \UNIX{} на Windows NT с
использованием компилятора \Cplusplus{} корпорации Microsoft. Именно
тогда я открыл для себя проект Cygwin. В качестве важного субпродукта
переноса я переработал свою систему сборки так, чтобы она работала на
операционной системе Windows NT. Новая система сборки также
поддерживала раздельные деревья каталогов исходных и бинарных файлов,
множество версий \UNIX{}, несколько вариантов графических оболочек,
автоматизированные ночные сборки и тесты, а также возможность создания
частичных рабочих копий и использования справочной сборки.

В 2000 году я стал одним из разработчиком системы управления
экспериментальными данными.  Основным языком разработки был язык
Java. Среда разработки в этом проекте очень сильно отличалась от тех,
в которых я работал много лет. Большая часть программистов имела опыт
программирования под Windows, и, похоже, язык Java был их первым
языком программирования. Среда сборки практически полностью состояла
из файла проекта, созданного коммерческой интегрированной средой
разработки для Java.  Не смотря на то, что файл проекта был тщательно
проверен, среда сборки редко работала <<из коробки>>, и программисты
часто сидели в своих кубиках, исправляя проблемы со сборкой системы.

Естественно, я начал писать собственную систему сборки, основанную на
\GNUmake{}, но меня ожидало неприятное удивление. Оказалось, что
большая часть разработчиков весьма неохотно пользовались командным
интерпретатором. Более того, многие не владели такими концепциями, как
переменные окружения и аргументы командной строки, или пониманием
инструментов, используемых для сборки программного обеспечения.
Интегрированные среды разработки прятали все эти детали от
пользователей. Чтобы учесть все эти тонкости, я усложнил свою систему
сборки, добавив развёрнутые описания ошибок, проверку предусловий,
управление конфигурацией машины разработчика и поддержку
интегрированных сред разработки.

В процессе реализации я прочитал руководство по GNU \GNUmake{}
несколько десятков раз. В поисках материала я наткнулся на второе
издание этой книги. Оно было наполнено очень полезной информацией, но,
к сожалению, опускало многие детали работы GNU \GNUmake{}. И это не
удивительно, учитывая возраст издания. Книга выдержала испытание
временем, но на момент 2003 года она нуждалась в обновлении. Третье
издание сфокусировано преимущественно на GNU \GNUmake{}. Как писал
Пол Смит (разработчик, поддерживающий GNU \GNUmake{}), <<не тратьте
силы, пытаясь написать переносимые \Makefile{}'ы, используйте
переносимую версию \GNUmake{}!>>.

%%--------------------------------------------------------------------
%% What's new in this edition
%%--------------------------------------------------------------------
\section*{Что нового в этом издании}
Практически весь материал книги обновлён. Я разделил его на три части.

Часть I, \textit{\nameref{part:basics}}, представляет достаточно
детальный обзор функциональности GNU \GNUmake{} и способов её
использования.

Глава~\ref{chap:simple_makefile}~--- это краткое введение в
\GNUmake{}, содержащее простой, но законченный пример. Она объясняет
базовые концепции \GNUmake{}, такие как цели и реквизиты, и описывает
синтаксис \Makefile{}'ов. Этого должно быть достаточно, чтобы вы
научились составлять свои первые \Makefile{}'ы.

В главе~\ref{chap:rules} обсуждаются структура и синтаксис правил.
Даётся детальное описание явных и шаблонных правил, в том числе
старомодных суффиксных правил. Здесь же обсуждаются специальные цели и
основы автоматического определения зависимостей.

Глава~\ref{chap:vars} содержит сведения о простых и рекурсивных
переменных. Здесь также обсуждается обработка \Makefile{}'ов, события,
приводящие к подстановке переменных и директивы для условной обработки
\Makefile{}'а.

В главе~\ref{chap:functions} рассматривается множество встроенных
функций GNU \GNUmake{}. Здесь же вводятся определяемые пользователем
функции и различные примеры их использования, начиная от тривиальных
иллюстраций и заканчивая сложными концепциями.

Глава~\ref{chap:commands} объясняет детали применения сценариев
сборки, покрывая такие аспекты, как их синтаксический разбор и
выполнение. Здесь же обсуждаются модификаторы команд, проверка кодов
возврата и окружение. Мы исследуем проблемы конечности длины командной
строки и способы их преодоления. На этом этапе вы будете знать все
возможности \GNUmake{}, используемые в этой книге.

Часть II, \textit{\nameref{part:advanced_topics}}, покрывает такие
большие темы, как использование \GNUmake{} в больших проектах,
переносимость и отладка.

В главе~\ref{chap:managing_large_proj} обсуждаются многие проблемы,
возникающие при сборке больших систем с помощью \GNUmake{}. Первой
темой будет использование рекурсивного вызова \GNUmake{} и способы
описания всего процесса сборки в единственном нерекурсивном
\Makefile'е. Также будут рассмотрены другие проблемы, возникающие в
больших системах, например, расположение файлов в файловой системе,
управление компонентами, автоматическая сборка и тестирование.

В главе~\ref{chap:portable_makefiles} обсуждаются вопросы
переносимости \Makefile'ов, преимущественно между различными версиями
\UNIX{} и Windows. В деталях рассматривается эмулятор среды
\UNIX{}~--- проект Cygwin~--- и связанные проблемы, возникающие при
использовании непереносимых инструментов и возможностей файловой
системы.

Глава~\ref{chap:c_and_cpp} содержит специфические примеры разделения
деревьев каталогов исходных и объектных файлов, а также способ
создания деревьев каталогов исходного кода с правами <<только для
чтения>>. Снова рассматривается анализ зависимостей, ударение делается
на решения, зависимые от языка. Эта глава вместе со следующей
неразрывно связана с вопросами, поднятыми в
главе~\ref{chap:managing_large_proj}.

В главе~\ref{chap:java} объясняется, как применить \GNUmake{} к среде
разработки приложений на языке \Java{}. Описываются техники управления
переменной \variable{CLASSPATH}, компиляции большого числа исходных
файлов, создания JAR-архивов и конструирования EJB.

Глава 10 начинается с обзора производительности некоторых операций
\GNUmake{}, дающего почву для рассуждений о производительности
\Makefile{}'ов. Обсуждаются техники обнаружения и устранения
критических участков кода, влияющих на производительность системы в
целом. Детально описывается возможность параллельного выполнения
команд при помощи GNU \GNUmake{}.

В Главе 11 представлены два примера реальных \Makefile{}'ов.
Первый из них использовался при создании этой книги. Он интересен по
двум причинам. Во-первых, степень автоматизации действий в нём
чрезвычайно высока. Во-вторых, он демонстрирует применение \GNUmake{}
в нетрадиционной для него области. В качестве второго примера
используются выдержки из системы сборки ядра Linux 2.6~---
\utility{kbuild}.

Глава 12 погружает нас в искусство отладки \Makefile{}'ов. Здесь мы
увидим техники определения действий, незримо выполняемых \GNUmake{}, а
также способы облегчения разработки.

Часть III, \textit{Дополнения}, включает вспомогательный материал.

Приложение А предоставляет собой справочник по опциям командной строки
утилиты GNU \GNUmake{}

В Приложении Б указаны пределы использования GNU \GNUmake{} по двум
характеристикам: управлении структурами данных и вычислении
производительности.

Приложение В содержит текст лицензии \emph{GNU Free Documentation
  License}, под которой распространяется текст этой книги.

%%--------------------------------------------------------------------
%% Conventions used in this book
%%--------------------------------------------------------------------
\section*{Типографские соглашения}
В книге приняты следующие типографские соглашения:

{\flushleft \textit{Наклонный шрифт}\\[1em] означает новые термины,
URL, адреса электронной почты, имена и расширения файлов, пути в
файловой системе и имена каталогов.}

\begin{alltt}
Моноширинный шрифт
\end{alltt}
означает исходный код, команды интерпретатора, опции командной строки,
содержимое файлов и вывод команд.

\begin{alltt}
\textbf{Моноширинный полужирный шрифт}
\end{alltt}
означает команды или другой текст, который должен быть набран
пользователем.

\begin{alltt}
\emph{Моноширинный наклонный шрифт}
\end{alltt}
означает текст шаблона, который должен быть заменён пользователем.

%%--------------------------------------------------------------------
%% Using code examples
%%--------------------------------------------------------------------
\section*{Использование примеров кода}
Эта книга была создана, чтобы помогать вам выполнять вашу работу. В
целом, вы можете использовать код, приведённый в этой книге, в ваших
программах и документации. Вам не нужно связываться с издательством и
просить разрешения, пока Вы не решите воспроизвести значительный кусок
кода. Например, написание программы, использующей несколько отрывков
кода из книги не требует разрешения. Для продажи или распространения
электронных носителей, содержащих примеры из книги, \textit{требуется}
разрешение. Для ответа на вопросы других людей со ссылкой на эту книгу
не требуется разрешение. Для внедрения значительного числа примеров
кода из этой книги в документацию вашего продукта \textit{требуется}
разрешение.

Если вам кажется, что ваше использование примеров кода из это книги
выходит за рамки законного использования, свяжитесь с издательством.

%%--------------------------------------------------------------------
%% Acknowledgments
%%--------------------------------------------------------------------
\section*{Благодарности}
Я хотел бы поблагодарить Ричарда Столлмена за то, что он подарил мне
мечту и веру в её исполнение. Естественно, без Пола Смита GNU
\GNUmake{} не существовал бы сейчас в том виде, в котором мы
используем. Спасибо тебе, Пол.

Я хотел бы поблагодарить моего редактора, Энди Орама, за его
неоценимую помощь и энтузиазм.

Отдельной благодарности заслуживает компания Cimarron Software,
которая предоставила мне среду, воодушевившую меня начать этот
проект. Хотелось бы также поблагодарить компанию Realm Systems,
предоставившую мне среду, воодушевившую меня закончить проект. В
частности, я хотел бы поблагодарить Дуга Адамсона, Кэти Андерсон и
Питера Букмена.

Спасибо моим рецензентам, Симону Геррети, Джону Макдональду и Полу
Смиту, предоставившим много глубоких замечаний и поправившим много
досадных ошибок.

Следующие люди заслуживают благодарность за значительный вклад в эту
работу: Стив Байер, Ричард Богарт, Бет Кобб, Джули Дэйли, Дэвид
Джонсон, Эндрю Мортон, Ричард Паймнтел, Брайэн Стивенс и Линус
Торвальдс.  Большое спасибо группе заговорщиков, обеспечивших мне
стабильный и спокойный островок в бушующих штормами морях: Кристине
Дилэйни, Тони Ди Сера, Джону Мажору и Даниелю Ридинг.

Наконец, выражаю глубокую благодарность и любовь своей жене, Мэгги
Кэстен, и нашим детям, Вильяму и Джэймсу, за их поддержку,
воодушевление и любовь на протяжении последних шести месяцев.  Спасибо
за то, что вы были со мной.

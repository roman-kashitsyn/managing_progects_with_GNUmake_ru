%%--------------------------------------------------------------------
%% Reference trees and third-party jars
%%--------------------------------------------------------------------
\section{Справочные деревья и архивы сторонних %
разработчиков}

Для того, чтобы использовать единое разделяемое справочное дерево с
поддержкой создания разработчиками частичных рабочих копий, просто
настройте механизм ночных сборок, создающий \Java{}\hyp{}архивы
проекта, и включите эти архивы в \variable{CLASSPATH} компилятора.
После этого шага разработчики смогут сделать нужную им частичную
рабочую копию и инициировать процесс компиляции (в предположении, что
список исходных файлов создаётся динамически программой, подобной
\utility{find}). Когда компилятору \Java{} нужно будет найти символ,
определённый в отсутствующем исходном файле, компилятор произведёт
поиск, основываясь на значении \variable{CLASSPATH}, и обнаружит
соответствующие файлы классов в архиве.

Получение \Java{}\hyp{}архивов сторонних разработчиков из справочного
дерева реализуется также просто. Просто поместите пути к этим архивам
в переменную \variable{CLASSPATH}. Как уже было замечено, \Makefile{}
может быть очень полезным инструментом управления этим процессом.
Разумеется, функция \function{get\hyp{}file} может быть использована
для автоматического выбора стабильной или бета версии, локальных или
удалённых \Java{}\hyp{}архивов при помощи соответствующего определения
переменной \variable{JAR\_PATH}.

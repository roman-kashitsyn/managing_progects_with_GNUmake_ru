%%--------------------------------------------------------------------
%% Portability issues
%%--------------------------------------------------------------------
\section{Проблемы переносимости}

Проблемы переносимости может быть нелегко охарактеризовать, поскольку
они могут варьироваться от тотальной смены парадигмы (например,
отличие классической Mac OS от System V \UNIX{}) до исправлений
тривиальных ошибок (таких, как исправление кода возврата программы).
Тем не менее, ниже перечислены основные проблемы переносимости, с
которыми рано или поздно сталкивается любой \Makefile{}:

\begin{description}
%---------------------------------------------------------------------
% Program names
%---------------------------------------------------------------------
\item[Имена программ] \hfill \\
Довольно часто в различных платформах для программ, реализующих схожую
(или даже одинаковую) функциональность, используются различные имена.
Наиболее ярким примером являются имена компиляторов языков \Clang{} и 
\Cplusplus{} (например, \utility{cc} и \utility{xlc}). Также общим
является добавления префикса \filename{g} для GNU-версий программ,
установленных на не GNU системах (например, \utility{gmake},
\utility{gawk}). 

%---------------------------------------------------------------------
% Paths
%---------------------------------------------------------------------
\item[Пути] \hfill \\
Расположение файлов и программ варьируется от платформы к платформе. 
Например, в операционной системе Solaris каталогом X-сервера
является \filename{/usr/X}, в то время как на многих других системах
этим каталогом является \filename{/usr/X11R6}. К тому же, различие
между \filename{/bin}, \filename{/usr/bin}, \filename{/sbin} и
\filename{/usr/sbin} часто неясно при переходе от одной системе к
другой.

%---------------------------------------------------------------------
% Options
%---------------------------------------------------------------------
\item[Аргументы командной строки] \hfill \\
Аргументы командной строки программы могут отличаться, в частности при
использовании альтернативной реализации. Более того, если на какой-то
платформе отсутствует нужная вам программа (или присутствующая версия
этой программы вам не подходит), вам, возможно, придётся заменить эту
программу другой, использующей другие аргументы командной строки.

%---------------------------------------------------------------------
% Shell features
%---------------------------------------------------------------------
\item[Возможности интерпретатора] \hfill \\
По умолчанию \GNUmake{} выполняет командные сценарии с помощью
\filename{/bin/sh}, однако возможности различных реализаций
интерпретатора \utility{sh} варьируются от платформы к платформе. В
частности, интерпретаторы, выпущенные до принятия стандарта \POSIX{},
не имеют множества возможностей и не принимают синтаксис современных
интерпретаторов.

У Open Group есть очень полезная статья, описывающая различия между
интерпретаторами System V и \POSIX{}. Её можно найти по адресу
\filename{\url{http://www.unix-systems.org/whitepapers/shdiffs.html}}.
Те, кому интересны детали, смогут найти спецификацию командного
интерпретатора \POSIX{} по адресу
\filename{\url{http://www.opengroup.org/onlinepubs/
007904975/utilities/xcu_chap02.html}}

%---------------------------------------------------------------------
% Program behavior
%---------------------------------------------------------------------
\item[Поведение программ] \hfill \\
Переносимым \Makefile{}'ам приходится бороться с программам, которые
ведут себя по-разному на различных платформах. Это встречается
повсеместно, поскольку различные поставщики исправляют (и совершают)
ошибки и добавляют новые возможности. Существуют также обновления
программ, которые поставщик может включить или не включить в релиз.
Например, в 1987 году программа \utility{awk} сменила старший номер
версии. Тем не менее, даже спустя двадцать лет некоторые системы всё
ещё не используют новую версию в качестве стандартной программы
\utility{awk}.

%---------------------------------------------------------------------
% Operating system
%---------------------------------------------------------------------
\item[Операционная система] \hfill \\
Наконец, существуют проблемы переносимости, связанные с совершенно
различными операционными системами, например, Windows и \UNIX{}, Linux
и VMS.
\end{description}

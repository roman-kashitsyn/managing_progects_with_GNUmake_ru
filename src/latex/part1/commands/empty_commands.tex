%%--------------------------------------------------------------------
%% Empty commands
%%--------------------------------------------------------------------
\section{Пустые команды}
\label{sec:empty_commands}

\index{Команды!пустые}
\newword{Пустая команда}~-- это команда, которая не производит никаких
действий:

{\footnotesize
\begin{verbatim}
header.h: ;
\end{verbatim}
}

Вспомним, что за списком реквизитов цели может следовать точка с
запятой и команда. Здесь используется только точка с запятой, что
означает, что команды не предполагаются. Вместо этого вы можете
поместить после определения цели строку, содержащую только один символ
табуляции, однако это будет невозможно прочитать. Пустые команды чаще
всего используются для предотвращения соответствия цели шаблонному
правилу и выполнения нежелательных команд.

Заметим, что в других версиях \GNUmake{} пустые цели иногда
используются в качестве абстрактных. В GNU \GNUmake{} следует
использовать специальную цель \target{.PHONY}, это безопасней и яснее.

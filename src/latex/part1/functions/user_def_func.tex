%%--------------------------------------------------------------------
%% User-defined functions
%%--------------------------------------------------------------------
\section{Функции, определяемые пользователем}
\label{sec:user_def_func}

\index{Функции!определяемые пользователем}
Хранение последовательностей команд в переменных открывает широкие
возможности для приложений. Например, вот небольшой макрос для
завершения процесса\footnote{
<<Зачем нам делать это в \Makefile{}'е?>>,~--- спросите вы. Например,
в Windows, открытие файла одним процессом делает его недоступным
для записи другому процессу. Пока я писал эту книгу, PDF файл часто
был заблокирован процессом программы Acrobat Reader, что не допускало
обновление PDF-файла с помощью \GNUmake{}. Поэтому я добавил эту
команду в несколько целей, чтобы завершать процесс Acrobat Reader
и открывать доступ к заблокированному файлу.
}:

{\footnotesize
\begin{verbatim}
AWK  := awk
KILL := kill

# $(kill-acroread)

define kill-acroread
  @ ps -W |                                           \
  $(AWK) 'BEGIN      { FIELDWIDTHS = "9 47 100" }     \
          /AcroRd32/ {                                \
                       print "Killing " $$3;          \
                       system( "$(KILL) -f " $$1 )    \
                     }'
endef
\end{verbatim}
}

\index{Cygwin}
Этот макрос был написан с использованием инструментов Cygwin\footnote{
Инструменты Cygwin~--- это порт многих стандартных утилит GNU и Linux
под Windows. Они включают набор компиляторов, X11R6, \utility{ssh} и
даже \utility{inetd}. Порт основывается на библиотеке, реализующей
системные вызовы \UNIX{} в терминах функций Win32 API. Это настоящий
инженерный шедевр, и я настоятельно рекомендую его вам. Загрузите
его с сайта \filename{\url{http://www.cygwin.com}}.
}, поэтому имена программ и опции команд \utility{ps} и \utility{kill}
нестандартны для \UNIX{}. Для завершения программы мы направляем вывод
программы \utility{ps} на вход \utility{awk}. Сценарий \utility{awk}
ищет процесс Acrobat Reader по его имени в системе Windows, и
завершает его в случае необходимости. Мы использовали возможности,
предоставляемые переменной \variable{FIELDWIDTH}, чтобы трактовать имя
команды со всеми аргументами как одно поле с точки зрения
\utility{awk}. Поэтому сценарий корректно напечатает полное имя
запущенной программы вместе с аргументами, даже если оно содержит
пробелы. Ссылки на поля в \utility{awk} обозначаются как
\variable{\${}1}, \variable{\${}2} и так далее.  Если обращения к
полям не экранировать, они могут быть рассмотрены как переменные
\GNUmake{}. Мы можем сообщить \GNUmake{}, что \variable{\${}n} не
нужно вычислять, с помощью экранирования символа доллара в
\variable{\${}n} вторым символом доллара, \variable{\${}\${}n}.  В
этом случае \GNUmake{} увидит два идущих подряд символа доллара,
уберёт лишний и передаст оставшийся в дочерний процесс командного
интерпретатора. 

Хороший макрос. К тому же, директива \directive{define} предохраняет
нас от дублирования кода в случае, если нам придётся использовать
макрос часто.  И всё же он далёк от совершенства. Что если мы захотим
завершить произвольный процесс, не только процесс Acrobat Reader?
Придётся ли нам определить ещё один макрос и написать сценарий заново?
Нет!

Переменным и макросам можно передавать аргументы, таким образом,
каждое их вычисление может отличаться от предыдущего. Параметры
макроса доступны в его теле через позиционные переменные
\variable{\${}1}, \variable{\${}2} и так далее. В качестве параметра
нашей функции \function{kill-acroread} подходит шаблон поиска:

{\footnotesize
\begin{verbatim}
AWK        := awk
KILL       := kill
KILL_FLAGS := -f
PS         := ps
PS_FLAGS   := -W
PS_FIELDS  := "9 47 100"

# $(call kill-program,awk-pattern)
define kill-program
  @ $(PS) $(PS_FLAGS) |                                  \
  $(AWK) 'BEGIN { FIELDWIDTHS = $(PS_FIELDS) }           \
          /$1/  {                                        \
                  print "Killing " $$3;                  \
                  system( "$(KILL) $(KILL_FLAGS) " $$1 ) \
                }'
endef
\end{verbatim}
}

Мы заменили шаблон поиска для \utility{awk}, \command{/AcroRd32/},
обращением к параметру, \variable{\${}1}. Заметим, что существует
тонкое различие между параметром макроса \variable{\${}1} и обращением
к полю \utility{awk} \variable{\${}\${}1}.  Очень важно помнить, какая
имена программа нуждается в получении значения переменной. Пока мы
улучшали нашу функцию, мы также присвоили ей более подходящее имя и
заменили жёстко закодированные значения (зависящие от Cygwin)
переменными. Теперь мы имеем довольно переносимый макрос для
завершения процесса.

Давайте испытаем его на практике:

{\footnotesize
\begin{verbatim}
FOP        := org.apache.fop.apps.Fop
FOP_FLAGS  := -q
FOP_OUTPUT := > /dev/null

%.pdf: %.fo
    $(call kill-program,AcroRd32)
    $(JAVA) $(FOP) $(FOP_FLAGS) $< $@ $(FOP_OUTPUT)
\end{verbatim}
}

Это шаблонное правило завершает процесс Acrobat Reader, если он
запущен на выполнение, и конвертирует \filename{fo}-файлы (Formatting
Objects) в \filename{pdf}-файл вызовом процессора \utility{fop}
(\filename{\url{http://xml.apache.org/fop}}). Вычисление макроса или
переменной имеет следующий синтаксис:

{\footnotesize
\begin{alltt}
\$(call \emph{имя-макроса}[, \emph{аргумент}\subi{1} . . . ])
\end{alltt}
}

\index{Функции!встроенные!call@\function{call}}
Встроенная функция \function{call} вычисляет свой первый аргумент,
заменяя в нём все вхождения позиционных аргументов (\variable{\${}1},
\variable{\${}2} и так далее) остальными своими аргументами.  (На
самом деле, она вовсе не <<вызывает>> макрос в смысле передачи
управления, а производит специальное вычисление макроса).  На месте
шаблона \ItalicMono{имя-макроса} может быть любая переменная или
макрос (вспомним, что макросы~--- это просто переменные, в значениях
которых допускаются символы новой строки). Макрос или переменная даже
не обязательно должны содержать позиционные аргументы
(\variable{\${}n}), однако в этом случае использование функции
\function{call} не имеет особого смысла. Аргументы макроса, следующие
за параметром \ItalicMono{имя-макроса}, разделяются запятыми.

Заметим, что первый аргумент функции \function{call} является именем,
а не значением переменной (то есть не начинается со знака доллара).
Это довольно необычно. Существуют и другие встроенные функции,
\index{Функции!встроенные!origin@\function{origin}}
принимающие на вход имена переменных, например, \function{origin}.
Если вы окружите первый аргумент функции \function{call} знаком
доллара и скобочками, его значение будет вычислено и передано на вход
\function{call}.

Проверка аргументов в функции \function{call} минимальна.  На вход
может быть передано любое количество аргументов.  Если макрос
использует позиционный аргумент \variable{\${}n}, для которого в
обращении к \function{call} нет фактического параметра, позиционный
аргумент заменяется пустой строкой.  Если аргументов \function{call}
больше, чем позиционных  аргументов макроса, они никогда не будут
подставлены в его тело.

Если вы вызываете один макрос из другого, вам стоит быть осторожным,
поскольку порой GNU \GNUmake{} 3.80 ведёт себя довольно странно.
Функция \function{call} во время подстановки определяет свои аргументы
как обычные переменные \GNUmake{}. Таким образом, если один макрос
вызывает другой, есть возможность, что аргументы вызывающего макроса
будут видны вызываемому макросу:

{\footnotesize
\begin{verbatim}
define parent
  echo "вызывающий макрос имеет два аргумента: $1, $2"
  $(call child,$1)
endef

define child
  echo "вызываемый макрос имеет один аргумент: $1"
  echo "однако он также имеет доступ к аргументу "
  echo "вызывающего макроса: $2!"
endef

scoping_issue:
    @$(call parent,one,two)
\end{verbatim}
}

После запуска \GNUmake{} мы сможем наблюдать проблему с областью
видимости:

{\footnotesize
\begin{alltt}
\$ \textbf{make}

вызывающий макрос имеет два аргумента: one, two
вызываемый макрос имеет один аргумент: one
однако он также имеет доступ к аргументу
вызывающего макроса: two!
\end{alltt}
}

В версии 3.81 это было исправлено, так что аргумент~\variable{\${}2}
вызываемого макроса вычислится как пустая строка.

Мы проведём ещё много времени, составляя собственные функции, однако
нам нужно ещё немного знаний, чтобы делать по-настоящему интересные
вещи!

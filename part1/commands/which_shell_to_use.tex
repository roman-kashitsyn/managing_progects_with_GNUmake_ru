%%-------------------------------------------------------------------
%% Which shell to use
%%-------------------------------------------------------------------
\section{Выбор командного интерпретатора}
\label{sec:which_shell_to_use}

Когда \GNUmake{} требуется передать команду интерпретатору, он
использует \utility{/bin/sh}. Вы можете изменить интерпретатор,
выставив соответствующим образом значение переменной \variable{SHELL}.
Однако хорошенько подумайте перед тем, как это сделать. Обычно
назначением \GNUmake{} является предоставление для команды
разработчиков инструмента сборки системы из исходного кода. Довольно
легко создать \Makefile{}, не соответствующий этому назначению,
используя инструменты, не доступные для других участников процесса
разработки или строя предположения, для них не справедливые.
Использование любых интерпретаторов, отличных от \utility{/bin/sh},
считается дурным тоном для любого широко распространённого приложения
(доступного через ftp или открытый репозиторий cvs). Мы обсудим
вопросы переносимости более детально в
главе~\ref{chap:portable_makefiles}.

Однако и есть другой контекст использования \GNUmake{}. В закрытых
средах разработки часто все участники проекта работают на ограниченном
множестве машин и операционных систем. На самом деле это именно та
среда, в которой мне чаще всего приходилось работать. В этой ситуации
вы имеете полное право настроить среду, в которой будет работать
\GNUmake{}, по своему усмотрению. Следует лишь инструктировать всех
разработчиков в вопросах настройки среды и работы со сборками.

В подобных средах я предпочитаю открыто жертвовать переносимостью в
некоторых аспектах. Я уверен, что это может сделать процесс разработки
гораздо более гладким. Одной их таких жертв является замена
стандартного значения переменной \variable{SHELL} на
\utility{/usr/bin/bash}. \utility{bash}~--- это переносимый,
\POSIX{}-совместимый командный интерпретатор (отсюда следует, что он
включает в себя все возможности \utility{sh}), являющийся
интерпретатором по умолчанию для GNU/Linux. Причиной многих проблем с
переносимостью \Makefile{}'ов является использование непереносимых
конструкций в сценариях сборки. Решением этих проблем является явное
использование одного стандартного интерпретатора вместо употребления
лишь переносимого подмножества команд \utility{sh}. У Пола Смита,
разработчика, занимающегося поддержкой GNU \GNUmake{}, ести
веб-страница <<Правила Пола для \Makefile{}'ов>> (<<Paul's Rules of
Makefiles>>,
\filename{\url{http://make.paulandlesley.org/rules.html}}), на которой
он делает следующее замечание: <<Не тратьте силы, пытаясь написать
переносимые \Makefile{}'ы, используйте переносимую версию
\GNUmake{}!>> (<<Don't hassle with writing portable makefiles, use
portable make instead!>>). Я могу добавить следующее: <<Когда есть
возможность, не тратьте силы, пытаясь написать переносимый сценарий,
используйте переносимый командный интерпретатор (bash)>>.
\utility{bash} работает на большинстве операционных систем, включая
практически все варианты \UNIX{}, Windows, BeOS, Amiga и OS/2.

В оставшейся части книги я буду явно указывать на случаи использования
специфичных возможностей \utility{bash}.

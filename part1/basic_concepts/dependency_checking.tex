%%--------------------------------------------------------------------
%% Dependency checking
%%--------------------------------------------------------------------
\section{Разрешение зависимостей}

Как же \GNUmake{} решает, что делать? Давайте рассмотрим выполнение
предыдущего примера более детально и выясним это.

Сначала \GNUmake{} замечает, что командная строка не содержит
аргументов и решает собрать цель по умолчанию, т.е.
\filename{count\_words}. Затем выполняется проверка реквизитов, их
обнаруживается три: \filename{count\_words.o}, \filename{lexer.o} и
\filename{-lfl}.  Затем \GNUmake{} ищет способ собрать цель
\filename{count\_words.o} и находит правило для неё. Снова происходит
проверка реквизитов. \GNUmake{} замечает, что цель
\filename{count\_words.c} не имеет правила, и файл с таким именем
существует, поэтому производится трансформация
\filename{count\_words.c} в \filename{count\_words.o}, для
осуществления которой выполняется следующая команда:

{\footnotesize
\begin{verbatim}
gcc -c count_words.c
\end{verbatim}
}

Подобная цепь переходов от целей к реквизитам и рассмотрение
реквизитов в качестве целей~--- типичный механизм, при помощи которого
\GNUmake{} проводит анализ \Makefile{}'а и решает, какие команды
подлежат выполнению.

Следующим реквизитом, рассматриваемым \GNUmake{}, является
\filename{lexer.o}. Цепочка правил, подобная рассмотренной, ведёт к
файлу \filename{lexer.c}, ещё не существующему. Далее \GNUmake{}
находит правило получения \filename{lexer.c} из \filename{lexer.l},
\index{flex}
согласно которому запускается программа \utility{flex}.
\index{gcc}
Теперь \filename{lexer.c} существует и можно запускать \utility{gcc}.

\index{Библиотека}
Наконец, \GNUmake{} проверяет реквизит \filename{-lfl}. Опция
\command{-l} сообщает \utility{gcc}, что приложение использует
системную библиотеку. Имя библиотеки <<fl>> преобразуется согласно
правилам именования библиотек в \filename{libfl.a}. GNU \GNUmake{}
имеет поддержку этого синтаксиса. Когда находится реквизит вида
\filename{-l<NAME>}, \GNUmake{} ищет файл с именем
\filename{libNAME.so} в стандартных каталогах библиотек. Если поиск
заканчивается неудачей, производится повторный поиск по имени
\filename{libNAME.a}.  В нашем случае поиск успешен, \GNUmake{}
находит файл \filename{/usr/lib/libfl.a} и производит финальное
действие~--- компоновку.

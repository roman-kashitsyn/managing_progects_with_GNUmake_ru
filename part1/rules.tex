%%%-------------------------------------------------------------------
%%% Rules
%%%-------------------------------------------------------------------
\chapter{Правила}
\label{chap:rules}

В предыдущей главе мы рассмотрели несколько правил для компиляции и
компоновки программы подсчёта слов. Каждое из этих правил определяло
цель~--- файл, который требуется собрать. Каждая цель зависела от
множества реквизитов, которые также являлись файлами. Когда
требовалось обновить цель, \GNUmake{} выполнял сценарий сборки только
в том случае, если файлы реквизитов имели дату модификации более
позднюю, чем цель. Поскольку цель одного правила может быть реквизитом
другого, множество целей и реквизитов может быть представлено в форме
\index{Граф зависимостей}
\newword{графа зависимостей} (\newword{dependency graph}). Составление
и обработка графа зависимостей является основной работой \GNUmake{}.

\index{Правила!явные}
Поскольку правила так важны для \GNUmake{}, существует несколько их
разновидностей. \newword{Явные правила}, наподобие тех, что были
использованы в предыдущей главе, указывают на необходимость обновления
цели при модификации или отсутствии файлов-реквизитов. Правила этого
типа вы будете писать наиболее часто.
\index{Правила!шаблонные}
\newword{Шаблонные правила} используют подстановки (wildcards) вместо
явного указания имён файлов.  Это позволяет \GNUmake{} применять такие
правила каждый раз при соответствии имени цели некоторому шаблону.
\index{Правила!неявные}
\newword{Неявные правила}~--- это явные или суффиксные правила,
встроенные в базу данных правил \GNUmake{}. Наличие встроенной базы
данных правил упрощает написание \Makefile'ов, поскольку для многих
общих задач уже известны типы файлов, суффиксы и сценарии сборки
целей.
\index{Правила!шаблонные!статические}
\newword{Статические шаблонные правила} отличаются от обычных
шаблонных правил тем, что могут быть применены только к определённому
списку целей.

GNU \GNUmake{} может быть использован как замена для многих других
версий \GNUmake{}. Он включает в себя множество возможностей,
сохранённых для поддержания обратной совместимости. Например,
\index{Правила!суффиксные}
\newword{Суффиксные правила} были реализованы в одной из первых версий
\GNUmake{} для написания общих правил. \GNUmake{} имеет поддержку
сиффиксных правил, однако они признаны устаревшими, поскольку могут
быть заменены более простыми и более \'{о}бщными шаблонными правилами.

%%--------------------------------------------------------------------
%% Explicit rules
%%--------------------------------------------------------------------
\section{Явные правила}
\label{sec:explicit_rules}

Чаще всего вам придётся писать именно явные правила, указывающие
некоторые файлы как цели и реквизиты. Правило может иметь более одной
цели. Это значит, что каждая из указанных целей имеет в точности то же
множество реквизитов, что и остальные цели. Если цели требуется
обновить, для каждой из них будет выполнен один и тот же сценарий. Вот
пример такого правила:

{\footnotesize
\begin{verbatim}
vpath.o variable.o: make.h config.h getopt.h gettext.h dep.h
\end{verbatim}
}

Это правило означает, что цели \filename{vpath.o} и
\filename{variable.o} зависят от одного и того же множества
заголовочных файлов. Оно имеет в точности тот же эффект, что и
следующая спецификация:

{\footnotesize
\begin{verbatim}
vpath.o: make.h config.h getopt.h gettext.h dep.h

variable.o: make.h config.h getopt.h gettext.h dep.h
\end{verbatim}
}

Обе цели собираются независимо. Если один из объектных файлов имеет
более раннюю дату модификации, чем один из указанных заголовочных
файлов, \GNUmake{} инициирует сборку и выполнит команды,
ассоциированные с правилом.

Правило не обязательно указывать полностью сразу. Каждый раз, когда
\GNUmake{} обнаруживает файл в качестве цели, он добавляет цель и
реквизиты в граф зависимостей. Если такая цель уже существовала в
графе, к записи о цели добавляются новые реквизиты. Одним из
элементарных применений этого свойства является разбиение длинной
строки на несколько более коротких для улучшения читабельности файла:

{\footnotesize
\begin{verbatim}
vpath.o: make.h config.h getopt.h gettext.h dep.h
vpath.o: filedef.h hash.h job.h commands.h variable.h vpath.h
\end{verbatim}
}

В наиболее сложных случаях список реквизитов может состоять из файлов, способы
обработки которых различны:

{\footnotesize
\begin{verbatim}
# Убедимся, что файл lexer.c существует до компиляции vpath.c
vpath.o: lexer.c

...

# Компилируем vpath.c с определёнными флагами
vpath.o: vpath.c
	$(COMPILE.c) $(RULE_FLAGS) $(OUTPUT_OPTION) $<

...

# Включаем файл зависимостей, составленный программой
include auto-generated-dependencies.d
\end{verbatim}
}

Первое правило декларирует, что цель \filename{vpath.o} должна быть
собрана заново при изменении файла \filename{lexer.c} (возможно,
генерация этого файла имеет некий побочный эффект). Правило также
может быть использовано для того, чтобы убедиться, что все реквизиты
существуют (или, в случае необходимости, обновлены) перед сборкой
цели. Нужно отметить двустороннюю сущность правил. При прямом чтении
правило означает, что, если файл \filename{lexer.c} изменился,
требуется выполнить действия по обновлению \filename{vpath.o}. При
чтении в обратном направлении правило означает, что если требуется
обновить \filename{vpath.o}, то нужно убедиться, что файл
\filename{lexer.c} существует. Это правило может быть помещено рядом с
остальными правилами, касающимися файла \filename{lexer.c}, чтобы
разработчики помнили об этой тонкой взаимосвязи. Далее, рассмотрим
правило компиляции \filename{vpath.o}. Сценарий сборки для этого
правила использует три переменных \GNUmake{}. Переменные будут
детально описаны позже, пока важно знать лишь то, что обращение к
переменной происходит с помощью знака доллара (\command{\$}), за
которым следует либо один символ, либо слово в круглых скобках.
Наконец, зависимости типа \filename{.o/.h} включаются из отдельного
файла, полученного при помощи внешней программы.

В качестве особого случая GNU \GNUmake{} поддерживает упрощённый
синтаксис для правил с одной командой.

{\footnotesize
\begin{alltt}
\emph{цель: ; команда}
\end{alltt}
}

На практике такие правила встречаются редко, однако всё же иногда они
могут быть полезны, особенно когда нужно сберечь место на экране
монитора или листе бумаги. 

%---------------------------------------------------------------------
% Wildcards
%---------------------------------------------------------------------
\subsection{Шаблоны}
Часто \Makefile'ы содержат огромное количество файлов. Для упрощения
работы с ними \GNUmake{} поддерживает шаблоны, идентичные шаблонам
командного интерпретатора \utility{Bourne shell}: \verb|~|, \verb|*|,
\verb|?|, \verb|[...]| и \verb|[^...]|.  Например, шаблону \verb|*.*|
соответствуют все файлы, содержащие в имени точку. Знак вопроса
означает один символ, а \verb|[...]|~--- класс символов. Для выбора
дополнения класса символов нужно использовать \verb|[^...]|. Знак
тильды (\verb|~|) может быть использован для обозначения домашнего
каталога текущего пользователя системы.  Если за тильдой следует имя
пользователя, будет подставлен домашний каталог указанного
пользователя. \GNUmake{} автоматически раскрывает шаблоны, когда они
встречаются в названиях целей, реквизитов или командных сценариях. В
другом контексте шаблоны могут быть раскрыты явным вызовом функции.
Шаблоны чрезвычайно полезны для написания более адаптивных
\Makefile{}'ов. Например, вместо того, чтобы явно перечислять все
файлы, входящие в состав исходного кода программы, вы можете
использовать шаблоны\footnote{В более серьёзных приложениях применение
шаблонов для выбора компилируемых файлов является плохой практикой,
поскольку может вызвать компоновку с посторонним опасным кодом. В
правилах удаления промежуточных файлов шаблоны могут быть фатальными
для проекта (прим. автора).}:

{\footnotesize
\begin{verbatim}
prog: *.c
	$(CC) -o $@ $^
\end{verbatim}
}

Однако очень важно быть осторожным с шаблонами, ими легко
злоупотребить. Рассмотрим пример:

{\footnotesize
\begin{verbatim}
*.o: constants.h
\end{verbatim}
}

Намерения очевидны: все объектные файлы зависят от заголовочного файла
\filename{constants.h}. Однако посмотрим, как раскроется шаблон в
каталоге, не содержащем объектных файлов:

{\footnotesize
\begin{verbatim}
: constants.h
\end{verbatim}
}

Это допустимое выражение \GNUmake{}, оно не вызовет ошибки, однако оно
также не выразит той зависимости, которую имел в виду пользователь.
Одним из корректных способов реализации этого правила является
использование шаблона для получения файлов с исходным кодом (которые,
как правило, присутствуют) и трансформация полученного списка в список
объектных файлов. Мы рассмотрим эту технику при обсуждении функций
в главе~{\ref{chap:functions}}.

Наконец, стоит отметить, что раскрытие шаблонов в тот момент, когда
они появляются в качестве целей или реквизитов, осуществляет
непосредственно \GNUmake{}. Однако раскрытие шаблонов в сценариях
происходит в дочернем процессе командного интерпретатора. Это может
быть важной деталью, поскольку \GNUmake{} раскрывает шаблоны во время
чтения \Makefile{}'а, а командный интерпретатор раскрывает их много
позже, во время непосредственного выполнения команд. Когда
производятся сложные манипуляции с файлами, результаты раскрытия
одинаковых шаблонов в разные моменты времени могут сильно отличаться.
Проблематичной может быть ситуация, когда некоторые файлы являются
результатом сборки, и \GNUmake{} не видит их во время обработки
\Makefile{}'а. К таким случаям нужно относится особенно осторожно.

%---------------------------------------------------------------------
% Phony targets
%---------------------------------------------------------------------
\subsection{Абстрактные цели}
\label{sec:phony_targets}
\index{Цели!абстрактные}
До этого момента все цели и реквизиты, рассматриваемые нами, были
файлами, которые нужно было создать или обновить. Хоть это и типичный
способ использования целей, часто бывает полезным представлять цель в
качестве метки для командного сценария. Например, ранее упоминалось,
что стандартной целью для многих \Makefile{}'ов является \target{all}.
Цели, не представляющие файлов, называют \newword{абстрактными целями}
(\newword{phony targets}). Ещё одной стандартной абстрактной целью
является \target{clean}:

{\footnotesize
\begin{verbatim}
clean:
    rm -f *.o lexer.c
\end{verbatim}
}

Абстрактные цели должны собираться всегда, потому что команды,
ассоциированные с правилом, не создают файл с именем цели.

Важно заметить, что \GNUmake{} не отличает абстрактных целей от
целей, являющимися файлами. Если по случайности файл с именем
абстрактной цели существует, \GNUmake{} будет ассоциировать этот файл
с абстрактной целью в графе зависимостей. Например, если в текущей
директории существует файл \filename{clean}, то запуск команды
\BoldMono{make clean} приведёт к появлению довольно неожиданного
сообщения:

{\footnotesize
\begin{alltt}
\$ \textbf{make clean}
make: `clean' is up to date.
\end{alltt}
}

Довольно часто абстрактные цели не имеют реквизитов; цель
\target{clean} всегда будет рассматриваться как не требующая
обновления, и ассоциированные с ней команды никогда не будут
выполнены.

Чтобы избежать этой проблемы, GNU \GNUmake{} имеет специальную цель,
\target{.PHONY}, позволяющую сообщить \GNUmake{}, что цель не является
настоящим файлом. Любая цель может быть объявлена как абстрактная с
путём включения её в список реквизитов цели \target{.PHONY}:

{\footnotesize
\begin{verbatim}
.PHONY: clean
clean: 
    rm -f *.o lexer.c
\end{verbatim}
}

Теперь \GNUmake{} всегда будет выполнять команды, ассоциированные с
целью \target{clean}, даже если файл с таким именем существует. В
добавок к пометке цели как требующей обновления, спецификация цели как
абстрактной сообщает \GNUmake{}, для этой цели не нужно использовать
стандартное правило получения файла цели из исходного кода. Это
позволяет \GNUmake{} провести оптимизацию обычного процесса поиска
правил для достижения более высокой производительности.

Довольно редко имеет смысл включать абстрактную цель в качестве
реквизита реального файла, поскольку это будет приводить к
безусловному обновлению цели. Указание же реквизитов абстрактных целей
довольно часто приносит пользу. Например, цель \target{all} имеет в
качестве реквизитов список программ, которые нужно собрать:

\begin{alltt}
.PHONY: all
all: bash bashbug
\end{alltt}

В предыдущем примере цель \target{all} собирает командный
интерпретатор \utility{bash} и инструмент отправки сообщений об
ошибках \utility{bashbug}.

Абстрактные цели могут рассматриваться как сценарии интерпретатора,
встроенные в \Makefile{}. Объявление абстрактной цели в качестве
реквизита другой цели вызовет запуск сценария, ассоциированного с
абстрактной целью, перед сборкой основной цели. Предположим, мы
ограничены в использовании дискового пространства, и хотим отобразить
количество доступного места на диске перед выполнением действий,
требующих значительных затрат дискового пространства. Одно из решений
демонстрирует следующий пример:

{\footnotesize
\begin{verbatim}
.PHONY: make-documentation
make-documentation:
    df -k . | awk 'NR == 2 { printf( "%d available\n", $$4 ) }'
    javadoc ...
\end{verbatim}
}

Проблема заключается в том, что нам может понадобиться указать
команды \utility{df} и \utility{awk} несколько раз для разных целей.
Это является проблемой с точки зрения поддержки, поскольку нам
придётся изменять каждое вхождение этих команд, если, например, в
другой системе формат выдачи данных утилиты \utility{df} отличается.
Поэтому более изящным решением является следующее:

{\footnotesize
\begin{verbatim}
.PHONY: make-documentation
make-documentation: df
    javadoc ...

.PHONY: df
df:
    df -k . | awk 'NR == 2 { printf( "%d available\n", \$\$4 ) }'
\end{verbatim}
}

Мы можем сообщить \GNUmake{} о необходимости вызова сценария,
ассоциированного с целью \utility{df}, перед созданием документации,
указав \utility{df} как реквизит цели \target{make-documentation}. Это
допустимо, поскольку \target{make-documentation} также является
абстрактной целью. Такой подход даёт нам ещё одно преимущество: теперь
мы можем легко использовать \utility{df} в других целях.

Существует много примеров удачного применения абстрактных целей.

Сообщения \GNUmake{} довольно трудны для чтения и отладки. На это есть
несколько причин: составление \Makefile{}'ов сверху\hyp{}вниз, в то
время как команды выполняются снизу\hyp{}вверх; кроме того, не
указывается, какая цель выполняется в данный момент. Чтобы исправить
ситуацию, полезно выводить сообщения о начале выполнения основных
целей. Абстрактные цели являются простым средством реализации этой
идеи. Ниже приведён отрывок из \Makefile{}'а командного интерпретатора
\utility{bash}:

{\footnotesize
\begin{verbatim}
$(Program): build_msg $(OBJECTS) $(BUILTINS_DEP) $(LIBDEP)
    $(RM) $@
    $(CC) $(LDFLAGS) -o $(Program) $(OBJECTS) $(LIBS)
    ls -l $(Program)
    size $(Program)

.PHONY: build_msg
build\_msg:
    @printf "#\n# Building $(Program)\n#\n"
\end{verbatim}
}

Поскольку \target{build\_msg} является абстрактной целью, сообщение
выводится непосредственно перед проверкой остальных реквизитов.  Если
бы сообщение о начале сборки было первой командой сценария сборки
\variable{\$(Program)}, то оно выводилось бы только после сборки всех
зависимых файлов. Также важно заметить, что, поскольку абстрактные цели
всегда помечены как требующие обновления, указание абстрактной цели
\target{build\_msg} в качестве реквизита \variable{\$(Program)}
вызовет безусловную сборку этой цели, даже если на самом деле
этого не требуется. В нашем случае это выглядит разумным, поскольку
основная работа заключается в компиляции исходных файлов в объектные,
а на финальном этапе будет производиться только компоновка.

Абстрактные цели также могут быть использованы для улучшения
<<пользовательского интерфейса>> \Makefile{}'а. Имена целей часто
содержат длинные стоки с путями к каталогам, дополнительные имена
компонентов (например, номера версий) и стандартные суффиксы. Это
может сделать указание имени нужной цели довольно неудобным занятием.
Этой проблемы можно избежать, добавив абстрактную цель указав в
качестве её реквизита имя нужной реальной цели.

Есть ряд абстрактных целей, являющихся более или менее стандартными.
Не смотря на то, что их имена являются лишь соглашением, эти цели
встречаются в большинстве \Makefile{}'ов. Список этих целей содержится
в Таблице~\ref{tab:std_phony_targets}.

\begin{table}
\begin{tabular}{|l|l|}
\hline
\textbf{Цель} & \textbf{Назначение}\\
\hline
\texttt{all} & Произвести сборку приложения.\\
\hline
\target{install} & Произвести установку собранного приложения.\\
\hline
\target{clean} & Удалить все бинарные файлы, полученные после сборки.\\
\hline
\target{distclean} & Удалить все файлы, не входящие в базовый дистрибутив.\\
\hline
\target{TAGS} & Создать таблицу тэгов для текстового редактора.\\
\hline
\target{info} & Создать файлы GNU info из файлов Texinfo.\\
\hline
\target{check} & Запустить все тесты, ассоциированные с приложением.\\
\hline
\end{tabular}
\caption{Стандартные абстрактные цели.}\label{tab:std_phony_targets}
\end{table}

Цель \target{TAGS} на самом деле не является абстрактной, поскольку
программы \utility{ctags} и \utility{etags} создают файл с именем
\filename{TAGS}. Эта цель включена в таблицу потому, что это
единственная стандартная реальная цель.

%---------------------------------------------------------------------
% Empty targets
%---------------------------------------------------------------------
\subsection{Пустые цели}
\index{Цели!пустые}
Пустые цели подобны абстрактным в том плане, что позволяют расширить
возможности \GNUmake{}. Абстрактные цели всегда требуют обновления и
вызывают сборку всех целей, \emph{зависимых} от абстрактной (т.е.
содержащих её в списке реквизитов). Предположим, однако, что у нас
есть команда, не ассоциированная с файлом, которую нужно выполнять
время от времени, причём зависимые цели не должны при этом
обновляться. Для этого мы можем воспользоваться целью, ассоциированной
с пустым файлом:

{\footnotesize
\begin{verbatim}
prog: size prog.o
    $(CC) $(LDFLAGS) -o $@ $^

size: prog.o
    size $^
    touch size
\end{verbatim}
}

Заметим, что правило \target{size} использует программу
\utility{touch} после своего завершения. Пустой файл используется
только для хранения времени последней модификации, и \GNUmake{} будет
выполнять правило \target{size} только в том случае, если файл
\filename{prog.o} подвергся изменению. Более того, спецификация
\target{size} как реквизита \filename{prog} будет вызывать обновление
\target{prog} только в том случае, если соответствующий объектный файл
изменялся.

\index{Переменные!автоматические!\${}?@\variable{\${}?}}
Пустые файлы бывают полезны в сочетании с автоматической переменной
\variable{\$?}. Мы обсудим автоматические переменные в разделе
<<\nameref{sec:automatic_vars}>>, но краткое описание этой переменной
здесь не повредит. Внутри сценария сборки каждого правила \GNUmake{}
определяет переменную \variable{\$?} как множество реквизитов, имеющих
более позднюю дату модификации, чем цель.  Вот пример правила,
печатающего имена всех файлов, изменившихся с момента последнего
выполнения команды \command{make print}:

{\footnotesize
\begin{verbatim}
print: *.[hc]
    lpr $?
    touch $@
\end{verbatim}
}

%%--------------------------------------------------------------------
%% Variables
%%--------------------------------------------------------------------
\section{Переменные}

Рассмотрим некоторые из тех переменных, которые мы использовали в
наших примерах. Самые простые из них имели следующий синтаксис:

{\footnotesize
\begin{alltt}
\emph{\$(имя-переменной)}
\end{alltt}
}

Эта запись означает, что мы хотим получить значение переменной с
именем \variable{имя\hyp{}переменной}. Переменные могут содержать
практически произвольный текст, а имена переменных допускают
использование большинства символов, включая знаки пунктуации.
Например, переменная, содержащая имя команды для компиляции исходного
кода на языке \Clang{}, имеет имя \variable{COMPILE.c}. Как правило,
для подстановки значения переменной её имя окружают символами
\command{\$(} и \command{)}. В случае, когда имя переменной состоит из
одного символа, скобки можно опускать.

Как правило, \Makefile{}'ы содержат много объявлений переменных. Кроме
того, существует множество переменных, определяемых непосредственно
\GNUmake{}. Некоторые из них предназначены для контроля пользователем
поведения \GNUmake{}, другие выставляются \GNUmake{} для
взаимодействия с пользовательским \Makefile{}'ом.

%---------------------------------------------------------------------
% Automatic variables
%---------------------------------------------------------------------
\subsection*{Автоматические переменные}
\label{sec:automatic_vars}

\index{Переменные!автоматические}
\newword{Автоматические переменные} вычисляются \GNUmake{} заново для
каждого исполняемого правила. Они предоставляют доступ к цели и списку
реквизитов, избавляя от необходимости явно указывать имена файлов.
Автоматические переменные полезны для избежания дублирования кода и
необходимы для написания шаблонных правил (их мы рассмотрим позже).

Существует семь автоматических переменных:

\begin{description}
%---------------------------------------------------------------------
% $@
%---------------------------------------------------------------------
\item[\variable{\$@}] \hfill \\
%---------------------------------------------------------------------
Имя файла цели правила. Если цель является элементом архива (archive
member), то <<\variable{\$@}>> обозначает имя файла архива.

%---------------------------------------------------------------------
% $%
%---------------------------------------------------------------------
\item[\variable{\$\%}] \hfill \\
%---------------------------------------------------------------------
Для целей, являющихся элементами архива, обозначает имя
элемента. Если цель не является элементом архива, то \variable{\$\%}
содержит пустое значение.

%---------------------------------------------------------------------
% $<
%---------------------------------------------------------------------
\item[\variable{\${}<}] \hfill \\
%---------------------------------------------------------------------
Имя первого реквизита в списке реквизитов.

%---------------------------------------------------------------------
% $<
%---------------------------------------------------------------------
\item[\variable{\${}?}] \hfill \\
%---------------------------------------------------------------------
Имена всех реквизитов, имеющих более позднюю дату модификации, чем
цель.

%---------------------------------------------------------------------
% $^
%---------------------------------------------------------------------
\item[\variable{\$\^}] \hfill \\
%---------------------------------------------------------------------
Имена всех реквизитов, разделённые пробелами. В списке отсутствуют
повторения элементов, поскольку для большинства задач (копирование,
компиляция и т.д.) повторения нежелательны.

%---------------------------------------------------------------------
% $+
%---------------------------------------------------------------------
\item[\variable{\$+}] \hfill \\
%---------------------------------------------------------------------
Подобно \variable{\$?}, содержит список имён реквизитов, разделённых
пробелами, с тем отличием, что может содержать повторения. Эта
переменная была введена для специфических ситуаций, таких как
компоновка, где повторение аргументов несёт особый смысл.

%---------------------------------------------------------------------
% $*
%---------------------------------------------------------------------
\item[\texttt{\$*}] \hfill \\
%---------------------------------------------------------------------
Основа имени файла цели. Как правило, основой является имя файла с
отброшенным суффиксом (мы рассмотрим вычисление основы в разделе
<<\nameref{sec:pattern_rules}>>). Использование этой переменной вне
шаблонных правил настоятельно не рекомендуется.
%---------------------------------------------------------------------
\end{description}

Кроме того, каждая из вышеперечисленных переменных имеет два варианта
для совместимости с другими версиями \GNUmake{}. Один из этих вариантов
возвращает название каталога, в которой находится соответствующий
файл. Этот вариант обозначается добавлением символа <<D>> к имени
переменной: \variable{\$(@D)}, \variable{\${}(<D)} и т.д. Второй
вариант возвращает только имена файлов без имени каталога, в котором
они находятся. Этот вариант обозначается добавлением символа <<F>> к
имени переменной: \variable{\$(@F)}, \variable{\$(<F)} и т.д.
Поскольку эти варианты имён содержат более одного символа, они должны
заключаться в круглые скобки. GNU \GNUmake{} предоставляет более
читабельные альтернативы в лице функций \function{dir} и
\function{notdir}. Мы обсудим эти функции в
главе~\ref{chap:functions}.

Вот пример нашего \Makefile{}'а, в котором явно указанные имена
заменены подходящими автоматическими переменными.

{\footnotesize
\begin{verbatim}
count_words: count_words.o counter.o lexer.o -lfl
    gcc $^ -o $@

count_words.o: count_words.c
    gcc -c $<

counter.o: counter.c
    gcc -c $<

lexer.o: lexer.c
    gcc -c $<

lexer.c: lexer.l
    flex -t $< > $@
\end{verbatim}
}

\input{./part1/rules/vpath.tex}
\input{./part1/rules/pattern_rules.tex}
\input{./part1/rules/implicit_rule_db.tex}
\input{./part1/rules/special_targets.tex}
%%--------------------------------------------------------------------
%% Automatic dependency generation
%%--------------------------------------------------------------------
\section{Автоматическое определение зависимостей}
\label{sec:auto_dep_gen}
Когда мы изменили нашу программу подсчёта слов так, чтобы часть
объявлений была описана в заголовочных файлах, мы, сами того не
замечая, добавили новую проблему. Мы описали зависимости между
объектными и заголовочными файлами в наш \Makefile{} самостоятельно. В
нашем случае сделать это было нетрудно, но в реальных программах (а не
в игрушечных примерах) это может быть весьма утомительным и
порождающим ошибки процессом. На самом деле, в большинстве программ
указание зависимостей практически невозможно, поскольку заголовочные
файлы могут включать другие заголовочные файлы, образуя сложное дерево
включений.
\index{Заголовочный файл}
Например, в моей системе один заголовочный файл \filename{stdio.h}
(наиболее часто используемый заголовочный файл стандартной библиотеки
языка \Clang{}) в общем счёте включает 15 других заголовочных файлов.
Разрешение подобных зависимостей вручную является практически
безнадёжным занятием. Однако неудавшаяся компиляция ведёт к часам
потраченного на отладку времени или, что ещё хуже, к проблемам в уже
выпущенном программном обеспечении. Что же нам делать?

К счастью, компьютеры весьма хорошо справляются с задачами поиска и
нахождения соответствий шаблону. Давайте используем программу для
определения зависимостей между исходными файлами, и даже записи этих
зависимостей в соответствии со стандартным синтаксисом \GNUmake{}. Как
вы, возможно, уже догадались, такая программа уже существует, по крайней
мере, для исходных файлов на \Clang{}/\Cplusplus{}.
\index{gcc}
\index{Опции!компилятора}
Компилятор \utility{gcc}, как и многие другие компиляторы, имеет опцию
для чтения исходных файлов и составления зависимостей для \GNUmake{}.
Например, так мы можем определить зависимости для \filename{stdio.h}

\begin{alltt}
\footnotesize
\$ \textbf{echo "#include <stdio.h>" > stdio.c}
\$ \textbf{gcc -M stdio.c}
stdio.o: stdio.c /usr/include/stdio.h /usr/include/\_ansi.h \textbackslash{}
/usr/include/newlib.h /usr/include/sys/config.h \textbackslash{}
/usr/include/machine/ieeefp.h /usr/include/cygwin/config.h \textbackslash{}
/usr/lib/gcc-lib/i686-pc-cygwin/3.2/include/stddef.h \textbackslash{}
/usr/lib/gcc-lib/i686-pc-cygwin/3.2/include/stdarg.h \textbackslash{}
/usr/include/sys/reent.h /usr/include/sys/\_types.h \textbackslash{}
/usr/include/sys/types.h /usr/include/machine/types.h \textbackslash{}
/usr/include/sys/features.h /usr/include/cygwin/types.h \textbackslash{}
/usr/include/sys/sysmacros.h /usr/include/stdint.h \textbackslash{}
/usr/include/sys/stdio.h
\end{alltt}

<<Отлично,>>~--- скажете вы,~---<<Теперь мне придётся запускать gcc,
открывать текстовый редактор и вставлять результаты работы компилятора
с ключом \command{-M} в свой \Makefile{}. Какой ужас.>>. И вы были бы
правы, если бы это была вся правда. Существует два стандартных способа
включения автоматически составленных зависимостей в \Makefile{}.
Первый, он же самый старый, заключается в добавлении комментария
наподобие следующего:

{\footnotesize
\begin{verbatim}
# Далее следуют автоматически составленные зависимости:
# НЕ РЕДАКТИРОВАТЬ
\end{verbatim}
}

{\flushleft
в конец \Makefile{}'а и написании сценария командного интерпретатора
для автоматического обновления этого раздела. Это, безусловно, гораздо
лучше ручного обновления, но всё ещё довольно неудобно. Второй метод
заключается в добавлении директивы include. Б\'{о}льшая часть версий
\GNUmake{} поддерживает эту директиву, и, безусловно, GNU \GNUmake{} в
их числе. Идея заключается в спецификации цели, с которой
ассоциированы действия по запуску \utility{gcc} с ключом \command{-M},
сохранении результатов в файле зависимостей и повторный запуск
\GNUmake{} с включением составленного файла зависимостей в основной
\Makefile{}. До появления GNU \GNUmake{} это делалось правилом
следующего вида:
}

{\footnotesize
\begin{verbatim}
depend: count\_words.c lexer.c counter.c
    $(CC) -M $(CPPFLAGS) $^ > $@
include depend
\end{verbatim}
}

Сначала вы запускаете \GNUmake{} с целью составить файл зависимостей,
и только после этого производите повторный пуск для сборки программы.
На момент появления этой возможности она выглядела неплохо, однако
часто люди добавляли или удаляли зависимости из исходного кода, забыв
заново составить файл зависимостей. Это становилось причиной
неправильной компиляции со всеми вытекающими неприятностями. GNU
\GNUmake{} решил эту неприятную проблему с помощью мощной
функциональности и довольно простого алгоритма. Рассмотрим сначала
алгоритм. Если мы составим для каждого исходного файла собственный
файл зависимостей, скажем, файл с расширением \filename{.d}, и добавим
этот файл в качестве цели к соответствующему правилу, то сможем
сообщить \GNUmake{}, что \filename{.d} файл нуждается в обновлении
(наряду с объектным файлом) при изменении исходного файла:

{\footnotesize
\begin{verbatim}
counter.o counter.d: src/counter.c include/counter.h include/lexer.h
\end{verbatim}
}

Составление этого правила может быть завершено шаблонным правилом и
довольно неуклюжим сценарием (взятым прямо из руководства по GNU
\GNUmake{}) \footnote{Этот довольно выразительный сценарий, по моему
мнению, всё же требует некоторого объяснения. Сначала мы используем
компилятор \Clang{} с опцией \command{-M} для создания временного
файла, содержащего список зависимостей цели. Имя временного файла
получается из названия цели \command{\${}@} и добавочного уникального
числового суффикса \command{.\${}\${}\${}\${}}. В командном
интерпретаторе \utility{sh} переменная \command{\${}\${}} содержит
идентификатор текущего запущенного процесса командного интерпретатора.
Поскольку этот идентификатор является уникальным, имя нашего
временного файла также получается уникальным.  Затем мы используем
\utility{sed} для добавления файла с расширением \filename{.d} в
качестве цели правила. Выражение \utility{sed} состоит из шаблона
поиска
\command{\textbackslash{}(\${}\textbackslash{})\textbackslash{}1.o[
:]*} и подстановки \command{\textbackslash{}1.o \${}@ :}, разделённых
запятыми. Шаблон поиска состоит из основы имени цели \command{\${}*},
заключенной в группу регулярного выражения
\command{\textbackslash{}(\textbackslash{})}, за которой следует
суффикс \command{.o}. После имени цели могут следовать пробелы или
двоеточия (\command{[ :]*}). Подстановка восстанавливает
первоначальную цель с помощью ссылки на первую группу регулярного
выражения с добавлением суффикса (\command{\textbackslash{}1.o}) и
добавляет файл зависимостей в качестве второй цели правила
(\command{\${}@}).}:

{\footnotesize
\begin{verbatim}
%.d: %.c
    $(CC) -M $(CPPFLAGS) $< > $@.$$$$;                  \
    sed 's,\($*\)\.o[ :]*,\1.o $@ : ,g' < $@.$$$$ > $@; \
    rm -f $@.$$$$
\end{verbatim}
}

Теперь рассмотрим вышеупомянутую функциональность. GNU \GNUmake{}
будет рассматривать каждый включаемый файл в качестве цели,
нуждающейся в обновлении.  Таким образом, когда мы будем упоминать
\filename{.d} файлы, \GNUmake{} автоматически попытается создать эти
файлы во время чтения \Makefile{}'а. Ниже представлен наш пример с
добавлением автоматического управления зависимостями:

{\footnotesize
\begin{verbatim}
VPATH    = src include
CPPFLAGS = -I include
SOURCES  = count_words.c \
           counter.c     \
           lexer.c 
count_words: counter.o lexer.o -lfl
count_words.o: counter.h
counter.o: counter.h lexer.h
lexer.o: lexer.h

include $(subst .c,.d,$(SOURCES))

%.d: %.c
    $(CC) -M $(CPPFLAGS) $< > $@.$$$$;                  \
    sed 's,\($*\)\.o[ :]*,\1.o $@ : ,g' < $@.$$$$ > $@; \
    rm -f $@.$$$$
\end{verbatim}
}

Директива включения должна появляться только после записанных вручную
правил, чтобы не подменить цель по умолчанию целью из включаемого
\index{Директивы!include@\directive{include}}
файла. Директива \directive{include} принимает в качестве аргумента
список файлов (чьи имена могут включать шаблоны). В предыдущем примере
\index{Функции!встроенные!substr@\function{substr}}
мы использовали встроенную функцию \GNUmake{} \function{substr} для
трансформации списка исходных файлов в список файлов зависимостей (мы
рассмотрим \function{substr} более подробно в разделе
<<\nameref{sec:str_func}>> главы~\ref{chap:functions}). Пока просто
примите к сведению, что мы используем эту функцию для замены строки
\filename{.c} на строку \filename{.d} в каждом слове списка
\variable{\${}(SOURCES)}.

\index{Опции!just-print@\command{-{}-just-print (-n)}}
Если теперь мы запустим \GNUmake{} с опцией
\command{-{}-just\hyp{}print}, то получим следующее:

\begin{alltt}
\footnotesize
\$ \textbf{make --just-print}
Makefile:13: count\_words.d: No such file or directory
Makefile:13: lexer.d: No such file or directory
Makefile:13: counter.d: No such file or directory
\verb#gcc -M -I include src/counter.c > counter.d.$$;       \#
\verb#sed 's,\(counter\)\.o[ :]*,\1.o counter.d : ,g'       \#
\verb#< counter.d.$$ > counter.d;                           \#
rm -f counter.d.\$\$
flex -t src/lexer.l > lexer.c
\verb#gcc -M -I include lexer.c > lexer.d.$$;           \#
\verb#sed 's,\(lexer\)\.o[ :]*,\1.o lexer.d : ,g'       \#
\verb#< lexer.d.$$ > lexer.d;                           \#
rm -f lexer.d.\$\$
\verb#gcc -M -I include src/count_words.c > count_words.d.$$; \#
\verb#sed 's,\(count_words\)\.o[ :]*,\1.o count_words.d : ,g' \#
\verb#< count_words.d.$$ count_words.d;                       \#
rm -f count\_words.d.\$\$
rm lexer.c
gcc -I include -c -o count\_words.o src/count\_words.c
gcc -I include -c -o counter.o src/counter.c
gcc -I include -c -o lexer.o lexer.c
gcc count\_words.o counter.o lexer.o /lib/libfl.a -o count\_words
\end{alltt}

Сначала \GNUmake{} выводит несколько предупреждений, с виду
напоминающих ошибки. Не стоит волноваться, это всего лишь
предупреждения. \GNUmake{} производит поиск файлов, указанных в
директиве \index{Директивы!sinclude@\directive{-include}}
\directive{include}, не находит их, и перед началом поиска правила для
создания этих файлов выводит предупреждение \command{No such file or
directory}. Эти предупреждения могут быть подавлены при помощи символа
\command{-}, добавленного перед директивой \directive{include}.
Следующие строки демонстрируют вызов \utility{gcc} с опцией
\command{-M} и запуск команды \utility{sed}.  Обратите внимание на то,
что \GNUmake{} должен вызвать \utility{flex} для создания
\filename{lexer.c}, удаляемый перед началом сборки цели по умолчанию.

\index{Автоматическое определение зависимостей}
Теперь у вас есть представление об автоматическом определении
зависимостей. Эта тема содержит ещё много интересных вопросов,
например, построение зависимостей для других языков программирования,
или вывод зависимостей в виде дерева. Мы вернёмся к этим темам во
второй части книги.

%%--------------------------------------------------------------------
%% Managing libraries
%%--------------------------------------------------------------------
\section{Управление библиотеками}
\label{sec:managing_libs}

\index{Библиотечный архив} \index{archive!library}
\index{Элемент архива} \index{archive!member}
\newword{Библиотечный архив} (\newword{archive library}), обычно
называемый просто библиотекой или архивом,~--- это специальный файл,
содержащий в себе другие файлы, именуемые \newword{элементами архива}
(\newword{archive members}). Например, стандартная библиотека языка
\Clang{} \filename{libc.a} содержит низкоуровневые функции. Библиотеки
используются настолько часто, что \GNUmake{} имеет специализированную
функциональность для создания, поддержки и компоновки архивов. Архивы
\index{Программы!ar@\utility{ar}}
создаются и модифицируются при помощи программы \utility{ar}.

Давайте вернёмся к нашему примеру. Мы можем модифицировать нашу
программу подсчёта слов, упаковав все её компоненты, пригодные для
повторного использования, в библиотеку.  Наша библиотека будет
состоять из двух файлов: \filename{counter.o} и \filename{lexer.o}.
Для создания библиотеки вызовем команду \filename{ar}:

{\footnotesize
\begin{alltt}
\$ \textbf{ar rv libcounter.a counter.o lexer.o}
a - counter.o
a - lexer.
\end{alltt}
}

Опции \command{rv} означают, что мы хотим заменить элементы библиотеки
указанными объектными файлами, и что \utility{ar} должен выводить отчёт
о своих действиях. Мы можем использовать действие замены даже в том
случае, если указанная библиотека не существует. Первым аргументом
после опций является имя библиотеки, за ним следуют имена объектных
файлов (некоторые версии \filename{ar} требуют опции \utility{c} в
случае, если библиотека ещё не существует, но GNU \utility{ar} не
требует этого). Две строки, следующие за вызовом команды \utility{ar},
являются отчётом о том, что объектные файлы были добавлены в
библиотеку.

Использование опции замены позволяет создавать и изменять архив
последовательно:

{\footnotesize
\begin{alltt}
\$ \textbf{ar rv libcounter.a counter.o}
r - counter.o
\$ \textbf{ar rv libcounter.a lexer.o}
r - lexer.o
\end{alltt}
}

Теперь \utility{ar} предваряет имена файлов символом "r". Это значит,
что файлы в архиве были заменены.

Библиотека может быть скомпонована в исполняемый файл несколькими
способами. Самый простой способ~--- просто указать имя библиотеки в
списке аргументов компилятора.  В свою очередь, компилятор или
компоновщик будут использовать расширение для определения типа каждого
из указанных в командной строке файлов:

{\footnotesize
\begin{verbatim}
cc count_words.o libcounter.a /lib/libfl.a -o count_words
\end{verbatim}
}

Компилятор \utility{cc} распознает два файла \filename{libcounter.a} и
\filename{/lib/libfl.a} как библиотеки и будет искать в них
недостающие символы.  Ещё одним способом ссылки на библиотеку является
опция \command{-l}:

{\footnotesize
\begin{verbatim}
cc count_words.o -lcounter -lfl -o count_words
\end{verbatim}
}

Как вы можете видеть, при использовании этой опции опускается префикс
и суффикс имени библиотеки. Опция \command{-l} делает командную строку
более компактной и удобочитаемой, однако, при использовании этой опции
вы получаете гораздо более весомое преимущество. Когда компилятор
\utility{cc} видит опцию \command{-l}, он \emph{ищет} библиотеку в
стандартных каталогах системных библиотек. Это избавляет программиста
от необходимости знать точный путь к файлу библиотеки и делает команду
компоновки более переносимой. К тому же, в системах, поддерживающих
разделяемые библиотеки (библиотеки с расширением \filename{.so} на
системах семейства \UNIX{}), компоновщик будет искать сначала
разделяемые библиотеки, и только если подходящей не обнаружено, будет
осуществлён поиск библиотечного архива. Такой подход позволяет
программам пользоваться преимуществами разделяемых библиотек без их
явной спецификации. Таково стандартное поведение компилятора и
компоновщика GNU. Старые компоновщики и компиляторы могут не
осуществлять такой оптимизации.

Список каталогов, в которых компилятор должен осуществлять поиск
библиотек, может быть изменён с помощью опции \command{-L},
указывающей список и порядок каталогов, в которых нужно искать
библиотеки. Эти каталоги будут добавлены в список прямо перед
системными каталогами библиотеки и будут использоваться для всех опций
\command{-l} в командной строке. На самом деле, компиляции в
предыдущем примере не завершится успехом, поскольку текущий каталог не
входит в список каталогов библиотек \filename{cc}. Мы можем решить эту
проблему добавлением текущего каталога в список как показано ниже:

{\footnotesize
\begin{verbatim}
cc count_words.o -L. -lcounter -lfl -o count_words
\end{verbatim}
}

Библиотеки вносят некоторые трудности в процесс сборки программ. Какие
возможности предоставляет \GNUmake{} для упрощения этого процесса? GNU
\GNUmake{} включает функциональность как по созданию библиотек, так и
использованию библиотек при компоновке. Давайте посмотрим, как это
работает.

%---------------------------------------------------------------------
% Creating and updating libraries
%---------------------------------------------------------------------
\subsection{Создаём и изменяем библиотеки}

Библиотеки фигурируют в \Makefile{}'е в качестве обычных файлов. Ниже
представлено простое правило для создания нашей библиотеки:

{\footnotesize
\begin{verbatim}
libcounter.a: counter.o lexer.o
    $(AR) $(ARFLAGS) $@ $^
\end{verbatim}
}

Это правило использует встроенные переменные \variable{AR} и
\variable{ARFLAGS}, содержащие имя программы \utility{ar} и
стандартные опции \command{rv} соответственно. Для спецификации файла
архива используется автоматическая переменная \variable{\$@}, а для
спецификации реквизитов~--- автоматическая переменная \variable{\$\^}.

Теперь, если вы укажите файл \filename{libcounter.a} в качестве
реквизита цели \target{count\_words}, \GNUmake{} обновит нашу
библиотеку перед компоновкой исполняемого файла. Обратите внимание на
одну деталь. \emph{Все} элементы архива будут замещены, даже если
среди них есть не изменявшиеся с момента последнего обновления архива
элементы. Чтобы не терять время впустую, мы можем написать более
подходящее правило:

{\footnotesize
\begin{verbatim}
libcounter.a: counter.o lexer.o
    $(AR) $(ARFLAGS) $@ $?
\end{verbatim}
}

Если вы используете \variable{\$?} вместо \variable{\$\^},
\GNUmake{} будет подставлять в список аргументов только те объектные
файлы, которые имеют более позднюю дату модификации, чем цель.

Можем ли мы ещё улучшить это правило? Может быть да, а может и нет.
\GNUmake{} имеет встроенную поддержку обновления отдельных файлов в
архиве, но прежде, чем мы вдадимся в эти детали, стоит сделать
несколько важных замечаний относительно такого подхода к работе с
библиотеками. Одна из основных задач \GNUmake{} состоит в том, чтобы
эффективно использовать время процессора и собирать только те файлы,
которые действительно в этом нуждаются. К сожалению, вызов
\utility{ar} для каждого элемента архива по отдельности при наличии
несколько десятков файлов занимает настолько много времени, что
перевешивает преимущество элегантного синтаксиса, рассмотренного
далее. Используя простой метод, представленный выше, мы можем вызвать
\utility{ar} один раз для всех изменившихся файлов и избежать
множества ненужных системных вызовов \filename{fork/exec}. Кроме того,
на многих системах использование ключа \command{r} при вызове
\utility{ar} очень неэффективно. На моём компьютере 1.9 GHz Pentium 4
создание большого архива, содержащего 14216 элементов общим размером
55 MB, занимает 4 минуты 24 секунды, в то время как замена одного
элемента в этом архиве требует 28 секунд. Таким образом, создание
архива заново будет более быстрой альтернативой замене элементов при
наличии более 10 (из 14216!) изменившихся файлов. В такой ситуации
более разумным подходом будет единовременное обновление архива с
использованием автоматической переменной \variable{\$?}. Для небольших
библиотек и более быстрых компьютеров нет причин отказываться от
элегантного подхода, описанного ниже, в пользу более простого, но и
более быстрого.

В GNU \GNUmake{} элемент архива может быть специфицирован при помощи
следующей нотации:

{\footnotesize
\begin{verbatim}
libgraphics.a(bitblt.o): bitblt.o
    $(AR) $(ARFLAGS) $@ $<
\end{verbatim}
}

Здесь \filename{libgraphics.a}~--- это имя библиотеки, а
\filename{bitblt.o} (сокращение от \newword{bit block transfer,
передача битовых блоков})~--- имя её элемента. Синтаксис
\filename{libgraphics.a(bitblt.o)} означает модуль, содержащийся в
библиотеке. Реквизитом для цели является сам объектный файл, а
командой~--- добавление этого файла в архив. Автоматическая переменная
\variable{\$<} используется для получения первого реквизита. На самом
деле существует встроенное шаблонное правило, предоставляющее в
точности ту же функциональность.

Когда мы соединим всё это воедино, наш \Makefile{} будет выглядеть
следующим образом:

{\footnotesize
\begin{verbatim}
VPATH    = src include
CPPFLAGS = -I include

count_words: libcounter.a /lib/libfl.a

libcounter.a: libcounter.a(lexer.o) libcounter.a(counter.o)

libcounter.a(lexer.o): lexer.o
    $(AR) $(ARFLAGS) $@ $<

libcounter.a(counter.o): counter.o
    $(AR) $(ARFLAGS) $@ $<

count_words.o: counter.h

counter.o: counter.h lexer.h

lexer.o: lexer.h
\end{verbatim}
}

При запуске \GNUmake{} выводит следующее:

{\footnotesize
\begin{alltt}
\$ \textbf{make}
gcc -I include -c -o count\_words.o src/count\_words.c
flex -t src/lexer.l > lexer.c
gcc -I include -c -o lexer.o lexer.c
ar rv libcounter.a lexer.o
ar: creating libcounter.a
a - lexer.o
gcc -I include -c -o counter.o src/counter.c
ar rv libcounter.a counter.o
a - counter.o
gcc count\_words.o libcounter.a /lib/libfl.a -o count\_words
rm lexer.c
\end{alltt}
}

Обратите внимание на правило обновления архива. Автоматическая
переменная \variable{\$@} приняла значение имени библиотеки, несмотря
на то, что имя цели в \Makefile{}'е было
\filename{libcounter.a(lexer.o)}.

Наконец, нужно отметить, что библиотечный архив включает индекс всех
символов, содержащихся в нём. Новые программы архиваторов, такие как
GNU \utility{ar}, обновляют этот индекс автоматически при добавлении в
архив нового символа. Более старые версии архиваторов могут этого не
делать. Для создания и обновления индекса архива используется
\index{Программы!runlib}
программа \utility{ranlib}. В системах со старой версией архиватора
должно использоваться правило следующего вида:

{\footnotesize
\begin{verbatim}
libcounter.a: libcounter.a(lexer.o) libcounter.a(counter.o)
    $(RANLIB) $@
\end{verbatim}
}

Вы также можете использовать альтернативный подход для больших
архивов:

{\footnotesize
\begin{verbatim}
libcounter.a: counter.o lexer.o
    $(RM) $@
    $(AR) $(ARFLGS) $@ $^
    $(RANLIB) $@
\end{verbatim}
}

Конечно, синтаксис управления элементами архива может использоваться с
применением встроенных правил. GNU \GNUmake{} содержит встроенные
правила обновления архивов. Если мы используем эти правила, наш
\Makefile{} будет выглядеть следующим образом:

{\footnotesize
\begin{verbatim}
VPATH    = src include
CPPFLAGS = -I include

count_words: libcounter.a -lfl

libcounter.a: libcounter.a(lexer.o) libcounter.a(counter.o)

count_words.o: counter.h

counter.o: counter.h lexer.h

lexer.o: lexer.h
\end{verbatim}
}

%---------------------------------------------------------------------
% Using libraries as prerequisites
%---------------------------------------------------------------------
\subsection{Использование библиотек в качестве реквизитов}

Когда библиотеки появляются в качестве реквизитов, они могут быть
обозначены с помощью расширения файла или опции \command{-l}. Если
указать имя файла библиотеки:

{\footnotesize
\begin{verbatim}
xpong: $(OBJECTS) /lib/X11/libX11.a /lib/X11/libXaw.a
    $(LINK) $^ -o $@
\end{verbatim}
}

{\noindent то компоновщик просто прочитает библиотечные файлы из
командой строки.  При использовании опции \command{-l} реквизиты вовсе
не выглядят обычными файлами:}

{\footnotesize
\begin{verbatim}
xpong: $(OBJECTS) -lX11 -lXaw
    $(LINK) $^ -o $@
\end{verbatim}
}

Когда в реквизитах используется форма \command{-l}, \GNUmake{}
производит поиск библиотеки (предпочитая разделяемую версию) и
подставляет абсолютный путь в переменные \variable{\$\^} и
\variable{\$?}. Одно из преимуществ такого подхода состоит в
возможности производить автоматический поиск библиотек даже в том
случае, если компоновщик в вашей системе не поддерживает такой
возможности. Другим преимуществом является возможность настройки
путей поиска \GNUmake{}, что позволяет вам производить поиск
собственных библиотек наравне с системными. В приведённом примере
первая форма (с использованием абсолютных путей) будет игнорировать
разделяемые библиотеки. При использовании же второй формы \GNUmake{}
будет знать, что разделяемые библиотеки более предпочтительны, поэтому
сначала произведёт поиск разделяемой версии \filename{X11}, и только в
случае неудачи будет выбрана статическая библиотека. Шаблоны
для распознавания имён библиотек хранятся в виде реквизитов специальной
\index{Цели!специальные!LIBPATTERNS@\target{.LIBPATTERNS}}
цели \target{.LIBPATTERNS} и могут быть настроены для различных
форматов имён библиотек.

К сожалению, есть одна неприятная мелочь. Если в какая-либо цель в
\Makefile{}'е специфицирует библиотеку, на неё нельзя ссылаться в
реквизитах с помощью опции \command{-l}.  Например, запуск \GNUmake{}
с таким \Makefile{}'ом:

{\footnotesize
\begin{verbatim}
count_words: count_words.o -lcounter -lfl
    $(CC) $^ -o $@ libcounter.a: libcounter.a(lexer.o)

libcounter.a(counter.o)
\end{verbatim}
}

{\noindent завершится неудачей со следующей ошибкой:}

{\footnotesize
\begin{verbatim}
No rule to make target `-lcounter', needed by `count_words'
\end{verbatim}
}

Причиной ошибки является то, что \GNUmake{} не совершил подстановку
\filename{libcounter.a} вместо \filename{-lcounter} и поиск подходящей
цели. Вместо этого был осуществлён обычный поиск библиотеки. Таким
образом, для библиотек, собранных в \GNUmake{}, должно указывается
непосредственно имя файла.

Компоновка больших программ без возникновения ошибок подобна искусству
чёрной магии. Компоновщик производит поиск библиотек в том порядке, в
каком они указаны в командной строке. Таким образом, если библиотека
\filename{A} содержит неопределённый символ, например, \textit{open},
определённый в библиотеке \filename{B}, то \filename{A} должна быть
указана в командной строке \emph{перед} \filename{B} (именно так,
\filename{A} требует \filename{B}). Иначе, когда компоновщик прочитает
\filename{A} и не найдёт определения символа \filename{open}, будет
слишком поздно возвращаться назад к \filename{B}. Компоновщик никогда не
осуществляет поиск в уже просмотренных библиотеках. Таким образом,
порядок появления библиотек в командной строке играет фундаментальное
значение.

Когда реквизиты цели сохраняются в переменных \variable{\$\^} и
\variable{\$?}, порядок их следования также сохраняется. Это
справедливо даже для реквизитов, размещённых в нескольких правилах. В
этом случае реквизиты каждого правила добавляются к списку реквизитов
в том порядке, в котором они появляются.

Родственной проблемой является проблема перекрёстных ссылок между
\index{Циклические ссылки}
библиотеками,также известных как \newword{циклические ссылки}
\index{Зацикливания}
(\newword{circular references}) или \newword{зацикливания}
(\newword{circularities}). Предположим, что после некоторой
модификации библиотека \filename{B} использует символ из \filename{A}.
Мы уже знаем, что \filename{A} должна быть указана до \filename{B}, но
теперь ещё и \filename{B} должно быть указана до \filename{A}.
Решением является ссылка на \filename{A} и до, и после ссылки на
\filename{B}: \command{-lA -lB -lA}. В больших и сложных программах
библиотеки часто должны повторяться подобным образом, иногда более
одного раза.

Такая ситуация ставит небольшую проблему при использовании \GNUmake{},
поскольку автоматические переменные, как правило, не содержат
дубликатов. Например, предположим, что нам нужно повторить библиотеку
в реквизитах для устранения циклических ссылок:

{\footnotesize
\begin{verbatim}
xpong: xpong.o libui.a libdynamics.a libui.a -lX11
    $(CC) $^ -o $@
\end{verbatim}
}

Этот список реквизитов после подстановки переменных будет выглядеть
следующим образом:

{\footnotesize
\begin{verbatim}
gcc xpong.o libui.a libdynamics.a /usr/lib/X11R6/libX11.a -o xpong
\end{verbatim}
}

Для подавления последствий такого поведения переменной \variable{\$\^}
в \GNUmake{} была добавлена переменная \variable{\$+}. Эта переменная
идентична \variable{\$\^} с той лишь разницей, что в списке реквизитов
сохраняются дубликаты. Используем \variable{\$+}:

{\footnotesize
\begin{verbatim}
xpong: xpong.o libui.a libdynamics.a libui.a -lX11
    $(CC) $+ -o $@
\end{verbatim}
}

Теперь список реквизитов породит следующую команду компоновки:

\begin{verbatim}
gcc xpong.o libui.a libdynamics.a libui.a \
/usr/lib/X11R6/libX11.a -o xpong
\end{verbatim}

%---------------------------------------------------------------------
% Double-colon rules
%---------------------------------------------------------------------
\subsection{Правила с двойным двоеточием}

\index{Правила!с двойным двоеточием}
Правила с двойным двоеточием~--- это реализация функциональности,
позволяющей собирать одну и ту же цель с помощью разных сценариев, в
зависимости от того, какое из подмножеств реквизитов было
модифицировано. Обычно если цель появляется более одного раза, все её
реквизиты соединяются в один список, сценарий сборки же для одной цели
может быть указан только один раз. При использовании же правил с
двойным двоеточием каждое появление цели рассматривается как отдельное
правило и обрабатывается индивидуально. Это значит, что для какой-то
определённой цели все правила должны быть одного типа: либо с одним
двоеточием, либо с двумя.

По-настоящему полезные применения этой возможности придумать довольно
сложно, поэтому давайте рассмотрим следующий искусственный пример:

{\footnotesize
\begin{verbatim}
file-list:: generate-list-script
    chmod +x $<
    generate-list-script $(files) > file-list

file-list:: $(files)
    generate-list-script $(files) > file-list
\end{verbatim}
}

Мы можем создать цель \target{file-list} двумя способами. Если
сценарий составления списка файлов изменился, то добавим файлу
сценария права на запуск и выполним его. Если изменились исходные
файлы, мы просто запускаем сценарий.  Несмотря на свою надуманность,
пример наглядно демонстрирует, как можно использовать эту
функциональность.

Мы рассмотрели большую часть функциональности \GNUmake{}, связанной с
правилами, которые, наряду с переменными и сценариями, составляют
самую сущность \GNUmake{}. Мы фокусировали внимание главным образом на
специфику синтаксиса и поведение различных возможностей, практически
не останавливаясь на способах их применения в более сложных ситуациях.
Это будет главным объектом нашего внимания во второй части книги. А
сейчас продолжим обсуждение переменных и команд.


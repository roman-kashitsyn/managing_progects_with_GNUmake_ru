%%-------------------------------------------------------------------
%% Automating builds and testing
%%-------------------------------------------------------------------
\section{Автоматические сборки и тестирование}

Как правило важно иметь возможность максимально возможной
автоматизации процесса сборки. Это позволит производить сборку
справочных деревьев каталогов по ночам, сохраняя дневное время
разработчиков. Это также позволяет разработчикам запускать сборки на
собственных машинах без предварительной подготовки.

Для программного обеспечения, находящегося в разработке, часто
возникает множество заявок на сборку различных версий различных
продуктов. Для человека, выполняющего эти заявки, возможность
запланировать несколько сборок и <<пойти прогуляться>> часто является
критичной для поддержки и выполнения заявок.

Автоматизированное тестирование создаёт дополнительные трудности.
Для управления процессом тестирования большинства консольных
приложений могут быть использованы простые сценарии. Для тестирования
консольных приложений, требующих взаимодействия с пользователем, можно
\index{dejaGnu}
использовать утилиту GNU \utility{dejaGnu}. Разумеется, каркасы,
\index{JUnit}
подобные JUnit (\filename{\url{http://www.junit.org}}), также
предоставляют поддержку модульного тестирования приложений, не
требующего графической среды.

Тестирование приложений с графическим пользовательским интерфейсом
готовит дополнительные проблемы.  Для систем, использующих X11, я с
успехом применял тестирование по расписанию с использованием
\index{Xvbf}
виртуального оконного буфера (virtual frame buffer), Xvfb. Для Windows
я не смог найти удовлетворительного решения для автоматизированного
тестирования. Все подходы основаны на сохранении тестовой учётной
записи зарегистрированной системе, а экрана~--- не заблокированным.

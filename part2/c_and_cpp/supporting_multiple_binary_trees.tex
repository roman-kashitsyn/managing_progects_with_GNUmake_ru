%%--------------------------------------------------------------------
%% Supporting multiple binary trees
%%--------------------------------------------------------------------
\section{Поддержка нескольких каталогов бинарных файлов}
\label{sec:supporting_multiple_binary_trees}

После реализации \Makefile{}'а, осуществляющего запись бинарных файлов
в отдельное дерево каталогов, реализовать поддержку множества таких
деревьев довольно просто. Для интерактивных сборок, инициируемых
разработчиками при помощи клавиатуры, требуется совсем мало
подготовки. Разработчик создаёт каталог для бинарных файлов, переходит
в него и вызывает \GNUmake{}, указав нужный \Makefile{}.

{\footnotesize
\begin{alltt}
\$ \textbf{mkdir -p ~/work/mp3\_player\_out}
\$ \textbf{cd ~/work/mp3\_player\_out}
\$ \textbf{make -f ~/work/mp3\_player/makefile}
\end{alltt}
}

Если процесс запуска сборки требует от разработчика больше участия, то
сценарий\hyp{}обёртка будет наилучшим решением. Этот сценарий может
также анализировать текущий каталог и выставлять соответствующим
образом переменные окружения, используемые в \Makefile{}'е (например,
\variable{BINARY\_DIR}).

{\footnotesize
\begin{verbatim}
#! /bin/bash
# Работаем в каталоге с исходными файлами.
curr=$PWD
export SOURCE_DIR=$curr
while [[ $SOURCE_DIR ]]
do
  if [[ -e $SOURCE_DIR/[Mm]akefile ]]
  then
    break;
  fi
  SOURCE_DIR=${SOURCE_DIR%/*}
done

# Если makefile не найден, выводим сообщение об ошибке.
if [[ ! $SOURCE_DIR ]]
then
  printf "run-make: Cannot find a makefile" > /dev/stderr
  exit 1
fi

# Если каталог для выходных файлов не задан, используем значение
# по умолчанию.
if [[ ! $BINARY_DIR ]]
then
  BINARY_DIR=${SOURCE_DIR}_out
fi

# Создаём каталог для бинарных файлов.
mkdir --parents $BINARY_DIR

# Запускаем make.
make --directory="$BINARY_DIR" "$@"
\end{verbatim}
}

Этот сценарий не очень сложен. Он производит поиск \Makefile{}'а в
текущем каталоге, и в случае неудачи поднимается вверх по дереву
каталогов, пока не найдёт \Makefile{}. Затем происходит проверка
наличия переменной окружения, содержащей каталог для бинарных файлов.
Если переменная не определена, ей присваивается значение по умолчанию,
получаемое добавлением суффикса <<\_out>> к имени каталога с исходными
файлами. Затем сценарий создаёт каталог для бинарных файлов и
осуществляет запуск \GNUmake{}.

Если осуществляются сборки для различных платформ, требуются методы
для определения нужной платформы. Наиболее простой подход требует от
разработчика определения переменной окружения для каждого типа
платформы и добавления условных директив, использующих эту переменную,
в \Makefile{} и в исходный код. Лучшим подходом является
автоматическое определение платформы на основании вывода программы
\utility{uname}.

{\footnotesize
\begin{verbatim}
space := $(empty) $(empty)
export MACHINE := $(subst $(space),-,$(shell uname -smo))
\end{verbatim}
}

Я считаю, что при автоматическом запуске сборки программой
\utility{cron} использование вспомогательного сценария командного
интерпретатора является более предпочтительным подходом, нежели прямой
вызов \GNUmake{}. Сценарий-обёртка предоставляет больше возможностей
для подготовки, обработки ошибок и завершения автоматизированной
сборки. Сценарий также является подходящим местом для определения
переменных и опций командной строки.

Наконец, если проект поддерживает фиксированное число деревьев
каталогов и платформ, вы можете использовать имена каталогов для
автоматического определения параметров текущей сборки. Например:

{\footnotesize
\begin{verbatim}
ALL_TREES := /builds/hp-386-windows-optimized \
             /builds/hp-386-windows-debug     \
             /builds/sgi-irix-optimzed        \
             /builds/sgi-irix-debug           \
             /builds/sun-solaris8-profiled    \
             /builds/sun-solaris8-debug

BINARY_DIR := $(foreach t,$(ALL_TREES),\
                $(filter $(ALL_TREES)/%,$(CURDIR)))

BUILD_TYPE := $(notdir $(subst -,/,$(BINARY_DIR)))

MACHINE_TYPE := $(strip              \
                  $(subst /,-,       \
                    $(patsubst %/,%, \
                      $(dir          \
                        $(subst -,/, \
                          $(notdir $(BINARY_DIR)))))))
\end{verbatim}
}

Переменная \variable{ALL\_TREES} содержит список всех возможных
каталогов бинарных файлов. Цикл \command{foreach} осуществляет
проверку соответствия текущего каталога одному из возможных каталогов
бинарных файлов, причём соответствовать может только один каталог. Как
только каталог определён, мы можем извлечь из имени каталога параметры
сборки (например, оптимизированная, отладочная или профилировочная).
Мы получаем последний компонент имени каталога, преобразуя набор слов,
разделённых запятой, в набор слов, разделённых слэшем, и извлекая
последнее слово этого набора при помощи функции \function{notdir}.
Извлечение названия целевой платформы осуществляется таким же
способом.

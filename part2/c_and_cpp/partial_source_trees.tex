%%--------------------------------------------------------------------
%% Partial source trees
%%--------------------------------------------------------------------
\section{Частичные рабочие копии}

В по-настоящему больших проектах простое создание рабочей копии и
поддержка исходного кода может быть тяжким бременем для разработчиков.
Если система состоит из большого числа модулей, и каждый разработчик
работает над небольшой её частью, создание полной рабочей копии и
компиляция всего проекта может быть непозволительной тратой времени.
Вместо этого можно использовать централизованные справочные
ночные сборки, служащие базой для заполнения недостающих файлов в
деревьях каталогов исходных и бинарных файлов разработчиков.

Реализация этого функционала потребует осуществления двух типов
поиска. Во-первых, если компилятору недостаёт заголовочного файла,
нужно дать ему инструкцию искать этот файл в справочном дереве
каталогов исходных файлов. Во-вторых, если \Makefile{}'у требуется
какая-то библиотека, нужно дать ему инструкцию искать её в справочном
дереве каталогов бинарных файлов. Для того, чтобы помочь компилятору
найти недостающий исходный код, мы можем просто указать дополнительную
опцию \command{-I} после аналогичной опции, специфицирующей локальные
каталоги заголовочных файлов. Чтобы помочь \GNUmake{} найти
библиотеки, мы можем указать дополнительные каталоги в директиве
\directive{vpath}.

{\footnotesize
\begin{verbatim}
SOURCE_DIR     := ../mp3_player
REF_SOURCE_DIR := /reftree/src/mp3_player
REF_BINARY_DIR := /binaries/mp3_player
...
include_dirs := lib $(SOURCE_DIR)/lib $(SOURCE_DIR)/include
CPPFLAGS     += $(addprefix -I ,$(include_dirs))                  \
                $(addprefix -I $(REF_SOURCE_DIR)/,$(include_dirs))
vpath %.h       $(include_dirs)                                   \
                $(addprefix $(REF_SOURCE_DIR)/,$(include_dirs))

vpath %.a       $(addprefix $(REF_BINARY_DIR)/lib/, codec db ui)
\end{verbatim}
}

Использование этого подхода предполагает, что наименьшей единицей,
которую можно извлечь из репозитория CVS, является библиотека или
программный модуль. В этом случае \GNUmake{} сможет пропустить
недостающие библиотеки и каталоги, если разработчик решил не делать их
рабочих копий. Когда будет нужно использовать эти библиотеки,
спецификация пути поиска поможет автоматически заполнить недостающие
файлы.

Переменная \variable{modules} нашего \Makefile{}'а содержит список
подкаталогов, в которых следует осуществлять поиск файлов
\filename{module.mk}. Если эти подкаталоги не содержатся в рабочей
копии, нужно удалить эти подкаталоги из списка. Кроме того, можно
присваивать значение переменной \variable{modules} при помощи функции 
\function{wildcard}:

{\footnotesize
\begin{verbatim}
modules := $(dir $(wildcard lib/*/module.mk))
\end{verbatim}
}

Это выражение вернёт список всех каталогов, содержащих файл
\filename{module.mk}. Заметьте, что благодаря использованию функции
\function{dir} имя каждого каталога будет оканчиваться слэшем.

\GNUmake{} также может осуществлять поддержку создания частичных
рабочих копий на уровне отдельных файлов, при сборке библиотеки
соединяя объектные файлы из локальной копии разработчика и, в случае
необходимости, из справочного дерева каталогов. Однако этот подход
имеет множество недостатков и, судя по моему опыту, разработчикам он
приносит больше вреда, чем пользы.

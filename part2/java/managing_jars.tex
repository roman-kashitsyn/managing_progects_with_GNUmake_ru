%%--------------------------------------------------------------------
%% Managing jars
%%--------------------------------------------------------------------
\section{Управление архивами \Java{}}

Сборка и управление \Java{}\hyp{}архивами поднимают проблемы, отличные
от тех, с которыми мы сталкивались при сборке библиотек
\Clang{}/\Cplusplus{}. На это есть три причины. Во\hyp{}первых,
элементы \Java{}\hyp{}архива адресуются относительным путём, поэтому
точные имена файлов, передаваемых программе \utility{jar}, нужно
тщательно контролировать. Во\hyp{}вторых, в \Java{}\hyp{}сообществе
есть тенденция соединять архивы, чтобы всё приложение могло
размещаться в единственном архиве. Наконец, \Java{}\hyp{}архивы могут
содержать файлы, отличные от файлов классов, например, файл манифеста,
файлы свойств и XML\hyp{}файлы.

Базовая команда для создания \Java{}\hyp{}архива при помощи GNU
\GNUmake{} выглядит следующим образом:

{\footnotesize
\begin{verbatim}
JAR      := jar
JARFLAGS := -cf

$(FOO_JAR): реквизиты...
    $(JAR) $(JARFLAGS) $@ $^
\end{verbatim}
}

Программа \utility{jar} может принимать вместо имён файлов имена
каталогов, в этом случае в архив будет помещено всё содержимое
указанных каталогов. Это может быть очень удобно, особенно при
использовании совместно с опцией \command{-C}, временно изменяющей
текущий каталог:

{\footnotesize
\begin{verbatim}
JAR      := jar
JARFLAGS := -cf

.PHONY: $(FOO_JAR)
$(FOO_JAR):
    $(JAR) $(JARFLAGS) $@ -C $(OUTPUT_DIR) com
\end{verbatim}
}

Здесь файл архива объявлен абстрактной целью. Однако при повторном
запуске \Makefile{}'а архив не будет создаваться заново, поскольку эта
цель не имеет реквизитов. Как и в случае команды \utility{ar},
описанной в одной из предыдущих глав, смысла в использовании флага
обновления архива, \command{-u}, практически нет, поскольку эта
операция занимает практически такое же (или даже большее) время, что и
операция создания нового архива.

\Java{}\hyp{}архив часто включает файл манифеста, в котором указан
поставщик, API и номер версии. Простой файл манифеста может выглядеть
следующим образом:

{\footnotesize
\begin{verbatim}
Name: JAR_NAME
Specification-Title: SPEC_NAME
Implementation-Version: IMPL_VERSION
Specification-Vendor: Generic Innovative Company, Inc.
\end{verbatim}
}

Этот файл содержит три переменных, \variable{JAR\_NAME},
\variable{SPEC\_NAME} и \variable{IMPL\_VERSION}, которые могут быть
заменены реальными значениями при создании архива с помощью
\utility{sed}, \utility{m4} или вашего любимого редактора потоков.
Ниже приведена функция для обработки файла манифеста:

{\footnotesize
\begin{verbatim}
MANIFEST_TEMPLATE := src/manifests/default.mf
TMP_JAR_DIR       := $(call make-temp-dir)
TMP_MANIFEST      := $(TMP_JAR_DIR)/manifest.mf

# $(call add-manifest, jar, jar-name, manifest-file-opt)
define add-manifest
  $(RM) $(dir $(TMP_MANIFEST))
  $(MKDIR) $(dir $(TMP_MANIFEST))
  m4 --define=NAME="$(notdir $2)"            \
     --define=IMPL_VERSION=$(VERSION_NUMBER) \
     --define=SPEC_VERSION=$(VERSION_NUMBER) \
     $(if $3,$3,$(MANIFEST_TEMPLATE))        \
     > $(TMP_MANIFEST)
  $(JAR) -ufm $1 $(TMP_MANIFEST)
  $(RM) $(dir $(TMP_MANIFEST))
endef
\end{verbatim}
}

Функция \function{add\hyp{}manifest} оперирует файлом манифеста
методом, подобным описанному выше. Сначала функция создаёт временный
каталог, затем производит подстановку переменных в шаблоне файла
манифеста. Затем функция обновляет архив и удаляет временный каталог.
Обратите внимание на то, что последний аргумент функции является
необязательным. Если путь к файлу манифеста не указан, функция
использует значение переменной \variable{MANIFEST\_TEMPLATE}.

В универсальном \Makefile{}'е эти операции привязаны к общей функции,
осуществляющей составление явного правила для создания
\Java{}\hyp{}архива:

{\footnotesize
\begin{verbatim}
# $(call make-jar,jar-variable-prefix)
define make-jar
  .PHONY: $1 $$($1_name)
  $1: $($1_name)
  $$($1_name):
      cd $(OUTPUT_DIR); \
      $(JAR) $(JARFLAGS) $$(notdir $$@) $$($1_packages)
      $$(call add-manifest, $$@, $$($1_name), $$($1_manifest))
endef
\end{verbatim}
}

Эта функция принимает один аргумент, префикс переменной \GNUmake{},
который идентифицирует набор переменных, описывающих четыре параметра
архива: имя цели, имя архива, пакеты архива и файл манифеста.
Например, для создания архива \filename{ui.jar} мы напишем следующее:

{\footnotesize
\begin{verbatim}
ui_jar_name     := $(OUTPUT_DIR)/lib/ui.jar
ui_jar_manifest := src/com/company/ui/manifest.mf
ui_jar_packages := src/com/company/ui \
                   src/com/company/lib

$(eval $(call make-jar,ui_jar))
\end{verbatim}
}

Используя композицию имён переменных, мы можем сократить
последовательность действий, выполняемых функцией, достигнув в тоже
время гибкой её реализации.

Если нам нужно создать много архивов, мы можем автоматизировать этот
процесс, поместив список имён архивов в переменную:

{\footnotesize
\begin{verbatim}
jar_list := server_jar ui_jar

.PHONY: jars $(jar_list)
jars: $(jar_list)

$(foreach j, $(jar_list),\
  $(eval $(call make-jar,$j)))
\end{verbatim}
}

В некоторых случаях нам может понадобиться распаковать содержимое
архива во временный каталог. Ниже представлен пример простой функции,
реализующей это требование:

{\footnotesize
\begin{verbatim}
# $(call burst-jar, jar-file, target-directory)
define burst-jar
  $(call make-dir,$2)
  cd $2; $(JAR) -xf $1
endef
\end{verbatim}
}

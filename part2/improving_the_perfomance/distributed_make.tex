%%--------------------------------------------------------------------
%% Distributed make
%%--------------------------------------------------------------------
\section{Распределённое выполнение \GNUmake{}}

GNU \GNUmake{} поддерживает малоизвестную (и практически не
тестированную) опцию для управления сборками, распределёнными среди
нескольких рабочих станций, соединённых сетью. Этот функционал основан
на библиотеке Customs, распространяемой с дистрибутивом
\utility{Pmake}. \utility{Pmake}~--- это альтернативная версия
\GNUmake{}, реализованная Адамом де Буром (Adam de Boor) в 1989 году
для операционной системы Sprite и всё ещё поддерживаемая Андреасом
Столке (Andreas Stolcke). Библиотека Customs помогает распределить
выполнение \GNUmake{} между множеством компьютеров. GNU \GNUmake{}
включает поддержку этой библиотеки начиная с версии 3.77.

Чтобы включить поддержку библиотеки Customs, вам нужно собрать
\GNUmake{} из исходного кода. Инструкцию по осуществлению этого
процесса можно найти в файле \filename{README.customs} в дистрибутиве
\GNUmake{}. Сначала вам нужно загрузить дистрибутив \utility{pmake}
(URL указан в инструкции), затем собрать \GNUmake{} с опцией
\command{-{}\hyp{}with\hyp{}customs}.

Сердцем библиотеки Customs является демон (daemon), запускаемый на
каждом узле распределённой вычислительной сети \GNUmake{}. Все узлы
должны иметь доступ к разделяемой файловой системе, предоставляемый,
например, NFS. Один экземпляр демона назначается управляющим.
Управляющий процесс назначает задачи участникам вычислительной сети.
Когда \GNUmake{} запускается с опцией \command{-{}\hyp{}jobs} больше
1, \GNUmake{} контактирует с управляющим процессом, вместе они
порождают задачи, распределяя их среди доступных узлов сети.

Библиотека Customs поддерживает множество возможностей. Узлы могут
группироваться по архитектуре и ранжироваться по производительности.
Узлам могут назначаться произвольные атрибуты, и задачи могут
назначаться на основании значений атрибутов и булевых операторов. В
добавок к этому, такие характеристики работы узлов, как время
простоя, свободное дисковое пространство, свободное пространство в
разделе подкачки, текущая средняя загрузка могут быть посчитаны во
время выполнения задач.

Если ваш проект реализован на \Clang{}, \Cplusplus{} или Objective-C
вам следует рассмотреть возможность применения программы
\utility{distcc} (\filename{\url{http://distcc.samba.org}}),
предназначенной для распределённой компиляции. \utility{distcc}
написана Мартином Пулом (Martin Pool) и другими программистами для
ускорения сборок проекта Samba. Это законченное робастное решение для
проектов, написанных на \Clang{}, \Cplusplus{} или Objective-C.
Для использования этого инструмента достаточно заменить компилятор
\Clang{} программой \utility{distcc}:

{\footnotesize
\begin{alltt}
\$ \textbf{make --jobs=8 CC=distcc}
\end{alltt}
}

Для каждой компиляции \utility{distcc} использует препроцессор
для обработки исходного кода, затем отправляет результат другим узлам
сети для компиляции. Наконец, удалённые узлы возвращают полученные
объектные файлы управляющему процессу. Этот подход устраняет
необходимость в разделяемой файловой системе, что существенно упрощает
установку и конфигурацию.

Множество рабочих узлов или \newword{добровольцев} можно указать
несколькими способами. Наиболее простым является перечисление
узлов\hyp{}добровольцев в переменной окружения перед запуском
\utility{distcc}:

{\footnotesize
\begin{alltt}
\$ \textbf{export DISTCC\_HOSTS='localhost wasatch oops'}
\end{alltt}
}

\utility{distcc} имеет много опций для управления списком удалённых
узлов, интеграцией с компилятором, управления компрессией, путями
поиска, а также обнаружения и исправления ошибок.

Ещё одним инструментом увеличения скорости компиляции является
программа \utility{ccache}, написанная руководителем проекта Samba
Эндрю Тридгеллом (Andrew Tridgell). Идея очень проста:
\utility{ccache} кэширует результаты предыдущих сборок. Перед
осуществлением компиляции осуществляется проверка, содержит ли кэш
нужные объектные файлы. Это не требует участие нескольких узлов сети,
не требуется даже существование сети. Автор сообщает о 5-10 кратном
ускорении основного процесса компиляции. Наиболее простым способом
использования этого инструмента является переопределение команды
компиляции через интерфейс командной строки:

{\footnotesize
\begin{alltt}
\$ \textbf{make CC='ccache gcc'}
\end{alltt}
}

\utility{ccache} можно использовать совместно с \utility{distcc} для
ещё большего ускорения процесса сборки. В добавок ко всему, оба этих
инструмента доступны в наборе инструментов Cygwin.

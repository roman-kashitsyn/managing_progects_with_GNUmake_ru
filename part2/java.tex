%%%-------------------------------------------------------------------
%%% Java
%%%-------------------------------------------------------------------
\chapter{\Java{}}
\label{chap:java}

\index{Интегрированные среды разработки}
Многие Java\hyp{}разработчики предпочитают использовать
интегрированные среды разработки (Integrated Development Environments,
IDE), например, Eclipse. У вас может возникнуть вопрос, зачем вам
нужно использовать \GNUmake{} в \Java{} проектах, если есть такие
известные альтернативы, как Ant и среды разработки \Java{}? Эта глава
содержит исследование значения \GNUmake{} в среде \Java{}, в
частности, в ней приводится универсальный \Makefile{}, который может
быть помещён с минимальными модификациями практически в любой
\Java{}\hyp{}проект для осуществления всех стандартных задач сборки.

Использование \GNUmake{} в совокупности с \Java{} поднимает несколько
проблем и предоставляет некоторые дополнительные возможности. Причиной
этого является сочетание трёх основных факторов: во-первых, компилятор
\Java{} работает очень быстро; во-вторых, стандартный компилятор
\Java{} поддерживает синтаксис \command{@fi\-le\-na\-me} для чтения
параметров командной строки из файла; в третьих, если в коде
\Java{}\hyp{}класса указан пакет, путь к \filename{.class}\hyp{}файлу
определяется однозначно.

Стандартный компилятор \Java{} работает очень быстро. Главной причиной
этого является принцип работы директивы \directive{import}. Подобно
директиве \directive{\#include} препроцессора языка \Clang{}, эта
директива используется для обеспечения доступа к внешним символам.
Однако вместо повторного чтения исходного кода, который затем
потребует повторного разбора и анализа, компилятор \Java{} считывает
файлы классов напрямую. Поскольку символы, определяемые в файле
класса, не могут измениться в процессе компиляции, компилятор
производит кэширование классов. Даже в случае проектов среднего
размера это означает, что компилятор \Java{} избавлен от необходимости
повторно считывать, разбирать и анализировать буквально миллионы строк
кода, с которыми пришлось бы работать компилятору языка \Clang{}.
Менее существенный прирост производительности достигается за счёт
свед\'{е}ния к минимуму оптимизаций, выполняемых большинством
компиляторов \Java{}. Вместо статической оптимизации предпочтение
отдаётся сложным оптимизациям времени выполнения (just-in-time, JIT),
осуществляемым виртуальной машиной \Java{} (\Java{} virtual machine,
JVM).

\index{Java!пакет}
Практически все крупные \Java{}\hyp{}проекты интенсивно используют
\newword{пакеты} (\newword{pack\-ages}). Каждый класс инкапсулируется в
пакет, определяющий область видимости символов, определённых в файле.
Имена пакетов имеют иерархическую структуру и неявно определяют
структуру файловой системы, предназначенную для их хранения. Например,
пакет \command{a.b.c} неявно определяет структуру каталогов
\filename{a/b/c}. Код, объявленный соответствующей директивой как
принадлежащий пакету \command{a.b.c}, будет скомпилирован в файлы
классов и помещён в каталог \filename{a/b/c}. Это означает, что
обычный алгоритм \GNUmake{}, отвечающий за ассоциацию бинарных файлов
с соответствующими исходными файлами, не будет работать правильно.
Однако это также означает, что нам больше не нужно указывать опцию
\command{-o} для спецификации каталога, предназначенного для
размещения объектного файла. Достаточно указать корень дерева
каталогов бинарных файлов, одинаковый для всех исходных файлов. Это, в
свою очередь, означает, что исходный код из различных каталогов может
быть скомпилирован одной и той же командой.

Все стандартные компиляторы \Java{} поддерживают синтаксис
\command{@fi\-le\-na\-me}, позволяющий считывать параметры командной
строки из файла. Это имеет большое значение в сочетании с функционалом
пакетов, поскольку позволяет производить компиляцию всего исходного
кода единственным вызовом компилятора. Такой подход даёт значительный
выигрыш в производительности, так как время, требуемое для загрузки и
работы компилятора, является значительной частью времени выполнения
сборки.

Итак, после составления соответствующей командной строки, компиляция
400\,000 строк \Java{}\hyp{}кода занимает около трёх минут при
использовании процессора Pentium 4 (2,5ГГц). Компиляция эквивалентного
по размеру приложения, написанного на \Cplusplus{}, потребует
нескольких часов.

%%--------------------------------------------------------------------
%% Alternatives to make
%%--------------------------------------------------------------------
\section{Альтернативы \GNUmake{}}

Как уже было замечено, сообщество \Java{}\hyp{}разработчиков с
энтузиазмом принимает новые технологии. Рассмотрим две из них, имеющие
отношение к \GNUmake{}~--- \utility{Ant} и интегрированные среды
разработки.

%---------------------------------------------------------------------
% Ant
%---------------------------------------------------------------------
\subsection{Ant}
\index{Ant@\utility{Ant}}
Сообщество \Java{}\hyp{}разработчиков очень активно и производит новые
инструменты с впечатляющей скоростью. Одним из таких инструментов
является \utility{Ant}~--- система сборки, призванная занять место
\GNUmake{} в процессе разработки \Java{}\hyp{}приложений. Как и
\GNUmake{}, \utility{Ant} использует файл спецификации для определения
целей и реквизитов проекта. В отличие от \GNUmake{}, \utility{Ant}
написан на языке \Java{} и принимает файлы спецификации в формате XML.

Чтобы дать вам представление о файле спецификации в формате XML,
приведу небольшую выдержку из файла сборки для \utility{Ant}:

{\footnotesize
\begin{verbatim}
<target name="build"
        depends="prepare, check_for_optional_packages"
        description="--> compiles the source code">
  <mkdir dir="${build.dir}"/>
  <mkdir dir="${build.classes}"/>
  <mkdir dir="${build.lib}"/>

  <javac srcdir="${java.dir}"
         destdir="${build.classes}"
         debug="${debug}"
         deprecation="${deprecation}"
         target="${javac.target}"
         optimize="${optimize}" >
    <classpath refid="classpath"/>
  </javac>
  
  ...

  <copy todir="${build.classes}">
    <fileset dir="${java.dir}">
      <include name="**/*.properties"/>
      <include name="**/*.dtd"/>
    </fileset>
  </copy>
</target>
\end{verbatim}
}

Как вы могли заметить, цель объявляется при помощи XML тега
\command{<target>}. Каждая цель имеет имя и список зависимостей,
указанных в атрибутах \command{name} и \command{depends}
\index{Ant!задачи}
соответственно. Действия, выполняемые \utility{Ant}, называются
\newword{задачами} (\newword{tasks}). Задачи реализованы на языке
\Java{} и привязаны к XML тегу. Например, задача создания каталога
специфицируется при помощи тега \command{<mkdir>} и вызывает
выполнение метода \command{Mkdir.execute}, который в конечном итоге
вызывает метод \command{File.mkdir}. Насколько это возможно, все
задачи реализуются средствами \Java{} API.

Эквивалентный файл сборки \GNUmake{} содержит следующий код:

{\footnotesize
\begin{verbatim}
# производит компиляцию исходного кода
build: $(all_javas) prepare check_for_optional_packages
    $(MKDIR) -p $(build.dir) $(build.classes) $(build.lib)
    $(JAVAC) -sourcepath $(java.dir) \
             -d $(build.classes)     \
             $(debug)                \
             $(deprecation)          \
             -target $(javac.target) \
             $(optimize)             \
             -classpath $(classpath) \
             @$<
    ...
    $(FIND) . \( -name '*.properties' -o -name '*.dtd' \) | \
    $(TAR) -c -f - -T - | $(TAR) -C $(build.classes) -x -f -

\end{verbatim}
}

Отрывок кода, приведённый выше, использует техники, которые мы ещё не
обсуждали. Пока удовлетворимся тем, что реквизит \target{all.javac}
содержит список всех \filename{java} файлов, которые нужно
скомпилировать. Задачи \utility{Ant} \command{<mkdir>},
\command{<javac>} и \command{<copy>} также осуществляют проверку
зависимостей. К примеру, если каталог уже существует, задача
\command{mkdir} не выполнит никаких действий. Более того, если файлы
\Java{}\hyp{}классов имеют более позднюю дату модификации, чем
соответствующие исходные файлы, компиляция не будет осуществляться.
Тем не менее, командный сценарий \GNUmake{} осуществляет по существу
такие же функции. \utility{Ant} включает общую задачу, именуемую
\command{<exec>}, используемую для запуска локальных программ.

\utility{Ant} использует искусный и оригинальный подход, однако, при
его использовании возникает несколько проблем, которые стоит
рассмотреть:

\begin{itemize}
%---------------------------------------------------------------------
\item Несмотря на то, что \utility{Ant} получил широкое
распространение в \Java{}\hyp{}сообществе, вне сообщества
\utility{Ant} практически не распространён. К тому же, сомнительно,
что его популярность когда-нибудь выйдет за пределы
\Java{}\hyp{}проектов (по причинам, перечисленным далее). \GNUmake{},
в свою очередь, успешно применяется во многих областях, включая
разработку программного обеспечения, обработку документов и
типографское дело, поддержку веб\hyp{}сайтов. Понимание \GNUmake{}
очень важно для любого, кому требуется работать в различных
программных системах.
%---------------------------------------------------------------------
\item Выбор XML как языка спецификаций вполне разумен для
\Java{}\hyp{}приложения. Однако читать и писать спецификации на языке
XML большинству людей не очень удобно. Хороший XML\hyp{}редактор может
быть нелегко найти или интегрировать с существующими инструментами
(если моя интегрированная среда разработки не содержит хорошего
XML\hyp{}редактора, мне придётся либо менять среду разработки, либо
искать такой редактор и использовать его отдельно). Как вы могли
видеть из предыдущего примера, \utility{Ant}\hyp{}диалект XML довольно
избыточен по сравнению с синтаксисом \GNUmake{}, и полон специфических
для XML особенностей.
%---------------------------------------------------------------------
\item В процессе работы с файлами \utility{Ant} вам нужно преодолевать
некоторую косвенность ваших спецификаций. Задача \utility{Ant}
\command{<mkdir>} те вызывает соответствующую программу
\utility{mkdir} вашей системы. Вместо этого вызывается метод
\command{mkdir()} класса \command{java.io.File}. Результатом вызова
может быть совсем не то, что вы ожидаете. По существу, любое
предположение программиста о поведении основных инструментов
\utility{Ant} должно быть проверено с привлечением документации по
\utility{Ant} или \Java{}, либо исходного кода \utility{Ant}. В
добавок, для вызова, к примеру, компилятора \Java{}, вам может
понадобиться разобраться в использовании десятка или более незнакомых
XML атрибутов, например, \command{srcdir}, \command{debug} и т.д., не
вошедших в руководство пользователя компилятора. В противоположность
этому \GNUmake{} совершенно прозрачен; как правило, вы можете просто
набирать команды прямо в интерпретаторе и следить за их поведением.
%---------------------------------------------------------------------
\item И всё же, несомненно, \utility{Ant} переносим, как и \GNUmake{}.
Как показано в главе~\ref{chap:portable_makefiles}, написание
переносимых \Makefile{}'ов, как и написание переносимых спецификаций
\utility{Ant}, требуют опыта и особых знаний. Программисты писали
переносимые \Makefile{}'ы два десятилетия. Более того, в документации
\utility{Ant} отмечается, что \utility{Ant} имеет проблемы
переносимости, связанные с символическими ссылками \UNIX{} и длинными
именами файлов в Windows, а MacOS X является единственной операционной
системой Apple, поддерживаемой \utility{Ant}, поддержка же других
платформ не гарантируется. К тому же, базовые операции наподобие
выставления флага исполняемости файлов не могут осуществляться при
помощи \Java{} API, для этого требуется вызов внешней программы.
Переносимость никогда не может быть простой или полной.
%---------------------------------------------------------------------
\item Программа \utility{Ant} не предоставляет подробного отчёта о
своих действиях. Поскольку задачи \utility{Ant} реализованы не в виде
командных сценариев, отображение действий, совершаемых этими задачами,
вызывает определённые трудности. Как правило, вывод состоит из
выражений на естественном языке, выдаваемых выражениями
\command{print}, добавленными автором задачи. Эти выражения не могут
быть выполнены пользователем в командной строке. В противоположность
этому, строки текста, отображаемые \GNUmake{} являются выражениями
интерпретатора и могут быть копированы из вывода и вставлены в
командный интерпретатор для повторного выполнения. Это означает, что
\utility{Ant} менее полезен для разработчиков, пытающихся понять
процесс сборки и способ работы инструментов, используемых в этом
процессе. Кроме того, это не даёт разработчику возможности повторно
использовать элементы этих задач экспромтом, при помощи клавиатуры.
%---------------------------------------------------------------------
\item Последняя и наиболее важная проблема заключается в том, что
\utility{Ant} сдвигает парадигмы осуществления сборок, призывая
использовать компилируемый язык программирования взамен
интерпретируемого. Задачи \utility{Ant} написаны на языке \Java{}.
Если какая-то задача не реализована или делает не то, что вы хотите,
вам нужно либо реализовать собственную задачу на \Java{}, либо
использовать задачу \command{<exec>} (разумеется, если вам приходится
часто использовать задачу \command{<exec>}, то гораздо проще
использовать \GNUmake{} с его макросами, функциями и более компактным
синтаксисом).

С другой стороны, интерпретируемые языки программирования были
изобретены для решения именно таких проблем. \GNUmake{} существует
около тридцати лет и может быть использован в большинстве сложных
ситуаций без расширения своей реализации. Разумеется, за эти тридцать
лет была реализована поддержка множества новых возможностей. Многие из
них задуманы и реализованы в GNU \GNUmake{}.
%---------------------------------------------------------------------
\end{itemize}

\utility{Ant} является замечательной программой, широко
распространённой в \Java{}\hyp{}сообществе. Тем не менее, прежде, чем
приступить к новому проекту, тщательно убедитесь, что \utility{Ant}
является подходящим инструментом для вашей среды разработки. Надеюсь,
эта глава докажет вам, что \GNUmake{} может быть успешно использован
для осуществления сборки вашего \Java{}\hyp{}проекта.

%---------------------------------------------------------------------
% Ant
%---------------------------------------------------------------------
\subsection{Интегрированные среды разработки}

Многие \Java{}\hyp{}разработчики используют интегрированные среды
разработки, совмещающие в единой (как правило, графической) среде
редактор, компилятор, отладчик и инструмент для навигации по исходному
коду. В качестве примеров можно привести такие проекты с открытым
исходным кодом, как Eclipse (\filename{\url{http://www.eclipse.org}})
и Emacs JDEE (\filename{\url{http://jdee.sunsite.dk}}), а также, если
рассматривать коммерческие разработки, Sun Java Studio
(\filename{\url{http://www.sun.com/software/sundev/jde}}) и JBuilder
(\filename{\url{http://www.borland.com/jbuilder}}). Эти среды, как
правило, имеют понятие процесса сборки проекта, заключающегося в
компиляции необходимых файлов и запуска приложения на выполнение.

Если интегрированная среда разработки поддерживает все эти операции,
зачем тогда нам рассматривать использование \GNUmake{}? Наиболее
очевидной причиной является переносимость. Если возникнет
необходимость осуществить сборку проекта на другой платформе, сборка
может закончится неудачей. Несмотря на то, что код \Java{} сам по себе
является переносимым, инструменты для работы с ним, как правило,
таковыми не являются. Например, конфигурационные файлы вашего проекта
могут включать списки путей в стиле \UNIX{} или Windows, это может
стать причиной ошибки при попытке запуска сборки под управлением
другой операционной системы. Второй причиной является тот факт, что
\GNUmake{} поддерживает автоматические сборки. Некоторые
интегрированные среды разработки поддерживают пакетные сборки, а
некоторые нет. Качество этой поддержки также варьируется. Наконец,
встроенная поддержка сборок часто бывает довольно ограниченной. Если
вы хотите реализовать собственную структуру каталогов, соответствующую
структуре релизов вашего проекта, интегрировать файлы помощи внешних
приложений, поддерживать автоматическое тестирование, ветвление и
параллельные треки разработки, скорее всего, вы обнаружите, что
встроенная поддержка сборок не подходит для ваших нужд.

По собственному опыту я могу судить, что интегрированные среды
разработки вполне подходят для небольших немасштабируемых приложений,
однако промышленные системы сборки требуют большей поддержки,
и \GNUmake{} может её обеспечить. Обычно я использую интегрированную
среду разработки для написания и отладки кода и составляю \Makefile{}
для промышленных сборок и релизов. Во время разработки я использую
интегрированную среду для компиляции проекта в состояние, пригодное
для отладки. Однако если я изменяю много файлов или модифицирую файлы,
являющиеся входными файлами для генератора кода, я запускаю
\Makefile{}. Интегрированная среда разработки, которую я использовал,
не имела соответствующей поддержки внешних программ, осуществляющих
генерацию кода. Обычно сборки, полученные с помощью интегрированной
среды, не подходят для поставок внутренним или внешним потребителям.
Для таких задач я использую \GNUmake{}.

\input{./part2/java/generic_java_makefile.tex}
\input{./part2/java/compiling_java.tex}
%%--------------------------------------------------------------------
%% Managing jars
%%--------------------------------------------------------------------
\section{Управление архивами \Java{}}

Сборка и управление \Java{}\hyp{}архивами поднимают проблемы, отличные
от тех, с которыми мы сталкивались при сборке библиотек
\Clang{}/\Cplusplus{}. На это есть три причины. Во\hyp{}первых,
элементы \Java{}\hyp{}архива адресуются относительным путём, поэтому
точные имена файлов, передаваемых программе \utility{jar}, нужно
тщательно контролировать. Во\hyp{}вторых, в \Java{}\hyp{}сообществе
есть тенденция соединять архивы, чтобы всё приложение могло
размещаться в единственном архиве. Наконец, \Java{}\hyp{}архивы могут
содержать файлы, отличные от файлов классов, например, файл манифеста,
файлы свойств и XML\hyp{}файлы.

Базовая команда для создания \Java{}\hyp{}архива при помощи GNU
\GNUmake{} выглядит следующим образом:

{\footnotesize
\begin{verbatim}
JAR      := jar
JARFLAGS := -cf

$(FOO_JAR): реквизиты...
    $(JAR) $(JARFLAGS) $@ $^
\end{verbatim}
}

Программа \utility{jar} может принимать вместо имён файлов имена
каталогов, в этом случае в архив будет помещено всё содержимое
указанных каталогов. Это может быть очень удобно, особенно при
использовании совместно с опцией \command{-C}, временно изменяющей
текущий каталог:

{\footnotesize
\begin{verbatim}
JAR      := jar
JARFLAGS := -cf

.PHONY: $(FOO_JAR)
$(FOO_JAR):
    $(JAR) $(JARFLAGS) $@ -C $(OUTPUT_DIR) com
\end{verbatim}
}

Здесь файл архива объявлен абстрактной целью. Однако при повторном
запуске \Makefile{}'а архив не будет создаваться заново, поскольку эта
цель не имеет реквизитов. Как и в случае команды \utility{ar},
описанной в одной из предыдущих глав, смысла в использовании флага
обновления архива, \command{-u}, практически нет, поскольку эта
операция занимает практически такое же (или даже большее) время, что и
операция создания нового архива.

\Java{}\hyp{}архив часто включает файл манифеста, в котором указан
поставщик, API и номер версии. Простой файл манифеста может выглядеть
следующим образом:

{\footnotesize
\begin{verbatim}
Name: JAR_NAME
Specification-Title: SPEC_NAME
Implementation-Version: IMPL_VERSION
Specification-Vendor: Generic Innovative Company, Inc.
\end{verbatim}
}

Этот файл содержит три переменных, \variable{JAR\_NAME},
\variable{SPEC\_NAME} и \variable{IMPL\_VERSION}, которые могут быть
заменены реальными значениями при создании архива с помощью
\utility{sed}, \utility{m4} или вашего любимого редактора потоков.
Ниже приведена функция для обработки файла манифеста:

{\footnotesize
\begin{verbatim}
MANIFEST_TEMPLATE := src/manifests/default.mf
TMP_JAR_DIR       := $(call make-temp-dir)
TMP_MANIFEST      := $(TMP_JAR_DIR)/manifest.mf

# $(call add-manifest, jar, jar-name, manifest-file-opt)
define add-manifest
  $(RM) $(dir $(TMP_MANIFEST))
  $(MKDIR) $(dir $(TMP_MANIFEST))
  m4 --define=NAME="$(notdir $2)"            \
     --define=IMPL_VERSION=$(VERSION_NUMBER) \
     --define=SPEC_VERSION=$(VERSION_NUMBER) \
     $(if $3,$3,$(MANIFEST_TEMPLATE))        \
     > $(TMP_MANIFEST)
  $(JAR) -ufm $1 $(TMP_MANIFEST)
  $(RM) $(dir $(TMP_MANIFEST))
endef
\end{verbatim}
}

Функция \function{add\hyp{}manifest} оперирует файлом манифеста
методом, подобным описанному выше. Сначала функция создаёт временный
каталог, затем производит подстановку переменных в шаблоне файла
манифеста. Затем функция обновляет архив и удаляет временный каталог.
Обратите внимание на то, что последний аргумент функции является
необязательным. Если путь к файлу манифеста не указан, функция
использует значение переменной \variable{MANIFEST\_TEMPLATE}.

В универсальном \Makefile{}'е эти операции привязаны к общей функции,
осуществляющей составление явного правила для создания
\Java{}\hyp{}архива:

{\footnotesize
\begin{verbatim}
# $(call make-jar,jar-variable-prefix)
define make-jar
  .PHONY: $1 $$($1_name)
  $1: $($1_name)
  $$($1_name):
      cd $(OUTPUT_DIR); \
      $(JAR) $(JARFLAGS) $$(notdir $$@) $$($1_packages)
      $$(call add-manifest, $$@, $$($1_name), $$($1_manifest))
endef
\end{verbatim}
}

Эта функция принимает один аргумент, префикс переменной \GNUmake{},
который идентифицирует набор переменных, описывающих четыре параметра
архива: имя цели, имя архива, пакеты архива и файл манифеста.
Например, для создания архива \filename{ui.jar} мы напишем следующее:

{\footnotesize
\begin{verbatim}
ui_jar_name     := $(OUTPUT_DIR)/lib/ui.jar
ui_jar_manifest := src/com/company/ui/manifest.mf
ui_jar_packages := src/com/company/ui \
                   src/com/company/lib

$(eval $(call make-jar,ui_jar))
\end{verbatim}
}

Используя композицию имён переменных, мы можем сократить
последовательность действий, выполняемых функцией, достигнув в тоже
время гибкой её реализации.

Если нам нужно создать много архивов, мы можем автоматизировать этот
процесс, поместив список имён архивов в переменную:

{\footnotesize
\begin{verbatim}
jar_list := server_jar ui_jar

.PHONY: jars $(jar_list)
jars: $(jar_list)

$(foreach j, $(jar_list),\
  $(eval $(call make-jar,$j)))
\end{verbatim}
}

В некоторых случаях нам может понадобиться распаковать содержимое
архива во временный каталог. Ниже представлен пример простой функции,
реализующей это требование:

{\footnotesize
\begin{verbatim}
# $(call burst-jar, jar-file, target-directory)
define burst-jar
  $(call make-dir,$2)
  cd $2; $(JAR) -xf $1
endef
\end{verbatim}
}

\input{./part2/java/reference_trees.tex}
%%--------------------------------------------------------------------
%% Enterprise JavaBeans
%%--------------------------------------------------------------------
\section{Enterprise JavaBeans}

\Java{}\hyp{}компоненты уровня предприятия (Enterprise
Java\-Beans\trademark{}, EJB)~--- это мощная техника инкапсуляции и
повторного использования бизнес\hyp{}логики, каркасом которой является
механизм удалённых вызовов методов (Remote Method Invocation, RMI).
EJB определяет \Java{}\hyp{}классы, используемые для реализации API
сервера, используемого, в конечном счёте, удалёнными клиентами. Эти
объекты и службы настраиваются при помощи специальных файлов в формате
XML.  После написания \Java{}\hyp{}класса и соответствующего ему
конфигурационного XML\hyp{}файла эти файлы нужно упаковать вместе в
\Java{}\hyp{}архив. Затем вызывается специальный EJB\hyp{}компилятор,
создающий код заглушек и связок, реализующих поддержку RPC.

Следующий код может быть добавлен в код универсального \Makefile{}'а
для предоставления поддержки EJB:

{\footnotesize
\begin{verbatim}
EJB_TMP_JAR = $(EJB_TMP_DIR)/temp.jar
META_INF    = $(EJB_TMP_DIR)/META-INF

# $(call compile-bean, jar-name,
#        bean-files-wildcard, manifest-name-opt)
define compile-bean
  $(eval EJB_TMP_DIR := $(shell mktemp -d \
                          $(TMPDIR)/compile-bean.XXXXXXXX))
  $(MKDIR) $(META_INF)
  $(if $(filter %.xml, $2),cp $(filter %.xml, $2) $(META_INF))
  cd $(OUTPUT_DIR) &&                     \
  $(JAR) -cf0 $(EJB_TMP_JAR)              \
         $(call jar-file-arg,$(META_INF)) \
         $(filter-out %.xml, $2)
  $(JVM) weblogic.ejbc $(EJB_TMP_JAR) $1
  $(call add-manifest,$(if $3,$3,$1),,)
  $(RM) $(EJB_TMP_DIR)
endef

# $(call jar-file-arg, jar-file)
jar-file-arg = -C "$(patsubst %/,%,$(dir $1))" $(notdir $1)
\end{verbatim}
}

Функция \function{compile\hyp{}bean} принимает три параметра: имя
\Java{}\hyp{}архива, который требуется создать, список файлов,
входящих в архив, и необязательный файл манифеста. Сначала при помощи
программы \utility{mktemp} создаётся пустой временный каталог, имя
каталога сохраняется в переменной \variable{EJB\_TMP\_DIR}. Поместив
присваивание этой переменной в функцию \function{eval}, мы получаем
гарантию того, что значение \variable{EJB\_TMP\_DIR} будет указывать
на новый временный каталог при каждом вычислении функции
\function{compile\hyp{}bean}. Поскольку функция
\function{compile\hyp{}bean} используется в командном сценарии,
она будет вычисляться только при выполнении сценария. Затем функция
осуществляет копирование всех XML файлов из списка
\variable{bean\hyp{}files\hyp{}wild\-card} в каталог
\filename{META\hyp{}INF}. Именно в этом каталоге хранятся
конфигурационные файлы EJB. После этого функция создаёт временный
\Java{}\hyp{}архив, используемый в качестве входа для
EJB\hyp{}компилятора. Функция \function{jar\hyp{}file\hyp{}arg}
преобразует имена вида \filename{dir1/dir2/dir3} к виду
\filename{-C dir1/dir2 dir3}, поэтому относительные имена файлов
архива корректны. Этот наиболее подходящий формат для передачи команде
\utility{jar} пути к каталогу \filename{META\hyp{}INF}. Поскольку
XML\hyp{}файлы, содержавшиеся в списке, уже скопированы в каталог
\filename{META\hyp{}INF}, мы отсеиваем их из списка аргументов команды
\utility{jar} при помощи функции \function{filter\hyp{}out}. После
сборки временного архива вызывается EJB\hyp{}компилятор Web\-Lo\-gic,
создающий результирующий архив. Затем к составленному архиву
добавляется файл манифеста. Последним действием является удаление
временного архива.

Способ использования новой функции очевиден:

{\footnotesize
\begin{verbatim}
bean_files = com/company/bean/FooInterface.class      \
             com/company/bean/FooHome.class           \
             src/com/company/bean/ejb-jar.xml         \
             src/com/company/bean/weblogic-ejb-jar.xml

.PHONY: ejb_jar $(EJB_JAR)
ejb_jar: $(EJB_JAR)
$(EJB_JAR):
    $(call compile-bean, $@, $(bean_files), weblogic.mf)
\end{verbatim}
}

Список \variable{bean\_files} немного необычен. Пути к файлам классов,
входящих в этот список, указаны относительно каталога
\filename{classes}, в то время как пути к XML\hyp{}файлам будут
вычисляться относительно каталога, в котором располагается
\Makefile{}.

Это всё замечательно, но что если ваш архив содержит много файлов?
Существует ли способ составить список файлов автоматически?
Разумеется:

{\footnotesize
\begin{verbatim}
src_dirs := $(SOURCE_DIR)/com/company/...

bean_files =                                          \
  $(patsubst $(SOURCE_DIR)/%,%,                       \
    $(addsuffix /*.class,                             \
      $(sort                                          \
        $(dir                                         \
          $(wildcard                                  \
            $(addsuffix /*Home.java,$(src_dirs)))))))

.PHONY: ejb_jar $(EJB_JAR)
ejb_jar: $(EJB_JAR)
$(EJB_JAR):
    $(call compile-bean, $@, $(bean_files), weblogic.mf)
\end{verbatim}
}

Этот код подразумевает, что список каталогов с исходными файлами
хранится в переменной \variable{src\_dirs} (в списке могут находится и
каталоги, не содержащие кода EJB\hyp{}компонентов), и что все файлы,
оканчивающиеся строкой \emph{Home.java}, идентифицируют пакеты,
содержащие код EJB\hyp{}компонентов. Выражение для определения
переменной \variable{bean\_files} сначала добавляет суффикс шаблона к
имени каждого каталога в списке, а затем вызывает функцию
\function{wild\-card} для нахождения всех файлов, имя которых
оканчивается строкой \emph{Home.java}. Имена файлов отбрасываются,
полученный список каталогов сортируется, дублирующиеся элементы
удаляются из списка. К каждому каталогу добавляется суффикс
\command{/*.class}, в результате командный интерпретатор заменит
шаблон списком соответствующих файлов классов. Наконец, от каждого
элемента списка отсекается префикс, содержащий имя каталога с
исходными файлами (поскольку такого подкаталога каталога
\filename{classes} не существует). Причиной использования шаблонов
командного интерпретатора вместо функции \function{wild\-card}
является тот факт, что \GNUmake{} не сможет гарантированно выполнить
поиск файлов, соответствующих шаблону, \emph{после} компиляции и
генерации файлов классов. Если \GNUmake{} вычислит функцию
\function{wild\-card} слишком рано, файлы не будут обнаружены, а кэш
содержимого каталогов помешает найти эти файлы позже. Применение же
функции \function{wild\-card} в каталоге с исходными файлами
совершенно безопасно, поскольку мы подразумеваем, что исходные файлы
не будут добавляться во время работы \GNUmake{}.

Предыдущий код будет работать в том случае, если у нас имеется
небольшое число архивов компонентов. Другой стиль разработки
подразумевает помещение каждого EJB\hyp{}компонента в собственный
\Java{}\hyp{}архив. Большие проекты могут содержать десятки архивов.
Для того, чтобы осуществлять автоматическую обработку этой ситуации,
нам нужно составить явное правило для каждого EJB\hyp{}архива. В нашем
примере исходный код EJB\hyp{}компонентов самодостаточен: каждый
компонент располагается в отдельном каталоге вместе с ассоциированным
XML\hyp{}файлом. Определить каталоги, содержащие EJB\hyp{}компоненты,
можно по наличию файлов, оканчивающихся строкой \emph{Session.java}.

Основной подход заключается в поиске EJB\hyp{}компонентов в каталогах
с исходным кодом, построении явного правила для каждого компонента и
записи этих правил в файл. Затем файл с правилами для
EJB\hyp{}компонентов включается в наш \Makefile{}. Создание файла с
правилами для компонентов вызывается через механизм управления
включаемыми файлами \GNUmake{}.

{\footnotesize
\begin{verbatim}
# session_jars - архивы EJB, адресованные относительным путём.
session_jars =
  $(subst .java,.jar,                       \
    $(wildcard                              \
      $(addsuffix /*Session.java, $(COMPILATION_DIRS))))

# EJBS - список всех EJB-архивов.
EJBS = $(addprefix $(TMP_DIR)/,$(notdir $(session_jars)))

# ejbs - Create all EJB jar files.
.PHONY: ejbs
ejbs: $(EJBS)
$(EJBS):
    $(call compile-bean,$@,$^,)
\end{verbatim}
}

С помощью вызова функции \function{wild\-card} со списком всех
каталогов с исходным кодом в качестве аргумента мы находим все файлы,
имя которых оканчивается на \emph{Session.java}. В нашем примере имя
архива образуется из имени найденного исходного файла с добавлением
расширения \filename{.jar}. Архивы будут помещаться во временный
каталог. Переменная \variable{EJBS} содержит список архивов,
адресованных относительным путём от корня дерева бинарных файлов.
Эти архивы являются целью, которую мы хотим обновить. Командным
сценарием является вызов функции \function{compile\hyp{}bean},
реализованной нами ранее. Фокус заключается в том, что список файлов
указан в качестве реквизита каждого архива. Давайте посмотрим, как
они будут создаваться.

{\footnotesize
\begin{verbatim}
-include $(OUTPUT_DIR)/ejb.d

# $(call ejb-rule, ejb-name)
ejb-rule = $(TMP_DIR)/$(notdir $1):            \
             $(addprefix $(OUTPUT_DIR)/,       \
               $(subst .java,.class,           \
                 $(wildcard $(dir $1)*.java))) \
             $(wildcard $(dir $1)*.xml)

# ejb.d - файл зависимостей EJB
$(OUTPUT_DIR)/ejb.d: Makefile
    @echo Вычисляю зависимости ejb...
    @for f in $(session_jars);       \
    do                               \
      echo "\$$(call ejb-rule,$$f)"; \
    done > $@
\end{verbatim}
}

Зависимости для каждого EJB\hyp{}архива записываются в файл
\filename{ejb.d}, включаемый в \Makefile{}. Когда \GNUmake{} первый
раз производит поиск этого файла, файл ещё не существует. Поэтому
\GNUmake{} вызывает правило для обновления включаемого файла. Это
правило записывает по одной строке, подобной следующей, для каждого
EJB\hyp{}архива:

{\footnotesize
\begin{verbatim}
$(call ejb-rule,src/com/company/foo/FooSession.jar)
\end{verbatim}
}

Результатом вычисления функции \function{ejb\hyp{}rule} является
имя целевого архива и списка реквизитов, как показано ниже:

{\footnotesize
\begin{verbatim}
classes/lib/FooSession.jar:                  \
    classes/com/company/foo/FooHome.jar      \
    classes/com/company/foo/FooInterface.jar \
    classes/com/company/foo/FooSession.jar   \
    src/com/company/foo/ejb-jar.xml          \
    src/com/company/foo/ejb-weblogic-jar.xml
\end{verbatim}
}

Таким образом, \GNUmake{} предоставляет возможность управлять довольно
большим количеством архивов без необходимости ручной поддержки набора
явных правил.

